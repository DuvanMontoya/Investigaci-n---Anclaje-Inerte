% !TeX program = pdflatex
\documentclass[pdflatex,sn-mathphys-num]{sn-jnl}

\usepackage[spanish,english]{babel}
\usepackage{amsmath,amssymb,amsfonts,amsthm,mathtools}
\usepackage{tikz}
\usetikzlibrary{arrows.meta,positioning,calc,decorations.pathreplacing}
\usepackage{subcaption}
\usepackage{ifthen}
\usepackage{float}
\usepackage[section]{placeins}
\usepackage{flafter}
\usepackage[nameinlink,noabbrev]{cleveref}
\usepackage{booktabs}
\usepackage{siunitx}

\graphicspath{{../../Figuras/Articulo/}{../../Figuras/Articulo/Extra/}{../../Figuras/Articulo/Diagnosticos/}}
\setcounter{tocdepth}{2}
\numberwithin{equation}{section}
\allowdisplaybreaks[2]
\sisetup{round-mode=places,round-precision=6}
\AtBeginDocument{\shorthandoff{<>}}

% --- Figure helper (compatible with existing body) ---
\newboolean{draftfigs}
\setboolean{draftfigs}{false}
\newcommand{\incfig}[2][]{%
    \ifthenelse{\boolean{draftfigs}}%
    {\fbox{\begin{minipage}{0.8\linewidth}\centering\texttt{FIGURE MISSING: #2}\\ \small(Draft mode enabled)\end{minipage}}}%
    {\IfFileExists{#2}%
        {\includegraphics[#1]{#2}}%
        {\fbox{\begin{minipage}{0.8\linewidth}\centering\texttt{MISSING FILE: #2}\end{minipage}}}%
    }%
}

% --- Theorem environments ---
\theoremstyle{plain}
\newtheorem{theorem}{Theorem}[section]
\newtheorem{proposition}[theorem]{Proposition}
\newtheorem{lemma}[theorem]{Lemma}
\newtheorem{corollary}[theorem]{Corollary}
\newtheorem{conjecture}[theorem]{Conjecture}
\theoremstyle{definition}
\newtheorem{definition}[theorem]{Definition}
\newtheorem{example}[theorem]{Example}
\theoremstyle{remark}
\newtheorem{remark}[theorem]{Remark}

% --- Shared notation ---
\newcommand{\Z}{\mathbb{Z}}
\newcommand{\Q}{\mathbb{Q}}
\newcommand{\R}{\mathbb{R}}
\newcommand{\C}{\mathbb{C}}
\newcommand{\N}{\mathbb{N}}
\newcommand{\F}{\mathbb{F}}
\newcommand{\vp}{v_p}
\newcommand{\Zp}{\Z_p}
\newcommand{\euler}{\gamma}
\newcommand{\Csbase}{C_{\mathrm{base}}}
\newcommand{\Csinf}{C_{\infty}}
\newcommand{\Cobs}{C_{\sigma,\Delta}}
\newcommand{\Ssing}{\mathfrak{S}_\Delta}
\newcommand{\Sodd}{\mathfrak{S}_{\Delta,\mathrm{odd}}}
\newcommand{\MR}[2]{M_{#1}(#2)}
\newcommand{\Prob}{\mathbb{P}}
\newcommand{\E}{\mathbb{E}}

\begin{document}

% Institutional cover page for submission package.
\begin{titlepage}
\centering
{\Large\bfseries Submission Cover Sheet\par}
\vspace{1.1cm}
{\large Target Journal: Inventiones Mathematicae\par}
\vspace{0.8cm}
{\LARGE\bfseries Inert Anchoring and Explicit $p$-adic Densities in the Family $R^2-c^2$\par}
\vspace{0.35cm}
{\large Deterministic renormalization and singular series in imaginary quadratic fields\par}
\vspace{1.0cm}
{\large Robert Duan Montoya Cardona\par}
\vspace{0.35cm}
{\normalsize Independent Researcher, Colombia\par}
\vspace{0.35cm}
{\normalsize Contact: \texttt{replace-with-author-email@example.com}\par}
\vfill
{\normalsize Prepared on \today\par}
\end{titlepage}

\title[Inert Anchoring for $R^2-c^2$]{Inert Anchoring and Explicit $p$-adic Densities in the Family $R^2-c^2$: Deterministic Renormalization and Singular Series in Imaginary Quadratic Fields}

\author*[1]{\fnm{Robert Duan} \sur{Montoya Cardona}}\email{replace-with-author-email@example.com}

\affil*[1]{\orgname{Independent Researcher}, \orgaddress{\city{Medellin}, \country{Colombia}}}

\abstract{We study the arithmetic family $M_R(c)=R^2-c^2$ for $1\le c\le R-1$, and the counting function
$N_\Delta(R)$ of values represented as norms from $K=\Q(\sqrt{\Delta})$ with fundamental $\Delta<0$.
The central mechanism is \emph{inert anchoring}: for an inert prime $p$, parity defects of $v_p(M_R(c))$
concentrate on the critical layer $v_p(c)=v_p(R)$ and therefore become deterministic rather than random.
This yields closed local formulas,
$\sigma_p^{(m)}=1-2/(p^m(p+1))$ for odd primes and
$\sigma_{2,\Delta}^{(m)}=1-1/(3\cdot 2^m)$ in the inert dyadic case.
Using these densities, we build a finite normalizer $H_\Delta(R)$ that isolates the local dependence on prime powers dividing $R$.
The normalized observable stabilizes toward a limit compatible with
$\Csinf(\Delta)=\Csbase(\Delta)\Ssing$, where
$\Csbase(\Delta)=2e^{-\gamma}L(1,\chi_\Delta)$.
Numerical evidence up to $R\le 10^7$ over all Heegner discriminants supports robust convergence and nontrivial local structure.
Under a standard distribution-level hypothesis for $M_R(c)$ in arithmetic progressions, we obtain
$N_\Delta(R)=\Csinf(\Delta)H_\Delta(R)R/\log R + O\!\left(R/(\log R)^{1+\delta}\right)$.}

\keywords{imaginary quadratic norms, $p$-adic densities, sieve constants, singular series, inert primes}
\pacs[MSC Classification]{11N32, 11R11, 11N36, 11D09}

\maketitle

\input{SubmitReady/Shared/body_main.tex}

\backmatter

\bmhead{Acknowledgements}
The author thanks the open-source mathematical software ecosystem used to generate all computational evidence.

\section*{Declarations}
\textbf{Funding:} Not applicable.

\textbf{Competing interests:} The author declares no competing interests.

\textbf{Ethics approval and consent to participate:} Not applicable.

\textbf{Consent for publication:} Not applicable.

\textbf{Data availability:} All generated data files are included in the project repository.

\textbf{Materials availability:} Not applicable.

\textbf{Code availability:} All scripts used in the analysis are included in the project repository.

\textbf{Author contribution:} Single-author paper; the author performed all mathematical, computational, and writing tasks.

\bibliography{SubmitReady/Shared/references}

\end{document}
