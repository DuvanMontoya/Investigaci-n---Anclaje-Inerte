% !TeX program = pdflatex
\documentclass[11pt,oneside]{article}

% ---------------------------------------------------------------
% Paquetes
% ---------------------------------------------------------------
\usepackage[utf8]{inputenc}
\usepackage[T1]{fontenc}
\usepackage[spanish,es-noquoting]{babel}
\usepackage{newtxtext,newtxmath}
\usepackage[final]{microtype}
\usepackage[a4paper,margin=1.0in]{geometry}

\let\Bbbk\relax
\let\openbox\relax
\usepackage{mathtools,amssymb,amsmath,amsthm}
\usepackage{bm}
\usepackage{enumitem}
\usepackage{booktabs}
\usepackage{siunitx}
\usepackage{graphicx}
\usepackage{tikz}
\usetikzlibrary{arrows.meta,positioning,calc,decorations.pathreplacing}
\graphicspath{{../../Figuras/Articulo/}{../../Figuras/Articulo/Extra/}{../../Figuras/Articulo/Diagnosticos/}}
\usepackage{ifthen}
\usepackage{float}
\usepackage[section]{placeins}
\usepackage{flafter}
\usepackage{titlesec}
\usepackage{titling}
\usepackage{tocloft}
\usepackage{fancyhdr}
\usepackage[font=small,labelfont=bf,labelsep=period,justification=justified,singlelinecheck=false]{caption}

% --- Robust Figure Handling ---
\newboolean{draftfigs}
\setboolean{draftfigs}{false} % Set to true if figures are missing

\newcommand{\incfig}[2][]{%
    \ifthenelse{\boolean{draftfigs}}%
    {\fbox{\begin{minipage}{0.8\linewidth}\centering\texttt{FIGURA NO ENCONTRADA: #2}\\ \small(Modo Draft Activado)\end{minipage}}}%
    {\IfFileExists{#2}%
        {\includegraphics[#1]{#2}}%
        {\fbox{\begin{minipage}{0.8\linewidth}\centering\texttt{ARCHIVO FALTANTE: #2}\end{minipage}}}%
    }%
}
\usepackage{subcaption}
\usepackage[bookmarks=false]{hyperref}
\usepackage[nameinlink,noabbrev]{cleveref}

\hypersetup{
  colorlinks=true,
  linkcolor=black,
  citecolor=black,
  urlcolor=black,
  pdftitle={Anclaje Inerte y Densidades p-adicas Explicitas en la familia R2-c2},
  pdfauthor={Robert Duan Montoya Cardona}
}
\sisetup{round-mode=places,round-precision=6}
\setlist[itemize]{itemsep=2pt,topsep=4pt,parsep=0pt,leftmargin=2em}
\setlength{\jot}{6pt}
\renewcommand{\arraystretch}{1.10}
\linespread{1.06}
\setlength{\parindent}{1.15em}
\setlength{\parskip}{2.5pt plus 0.8pt minus 0.5pt}
\setlength{\emergencystretch}{2em}
\captionsetup{width=0.94\linewidth}
\titleformat{\section}{\normalfont\Large\bfseries}{\thesection}{0.65em}{}
\titleformat{\subsection}{\normalfont\large\bfseries}{\thesubsection}{0.60em}{}
\titleformat{\subsubsection}{\normalfont\normalsize\bfseries}{\thesubsubsection}{0.60em}{}
\titlespacing*{\section}{0pt}{1.35\baselineskip}{0.65\baselineskip}
\titlespacing*{\subsection}{0pt}{1.05\baselineskip}{0.52\baselineskip}
\titlespacing*{\subsubsection}{0pt}{0.90\baselineskip}{0.40\baselineskip}
\raggedbottom
\setcounter{tocdepth}{2}
\renewcommand{\contentsname}{Índice}
\renewcommand{\cfttoctitlefont}{\hfill\Large\bfseries}
\renewcommand{\cftaftertoctitle}{\hfill}
\setlength{\cftbeforesecskip}{2pt}
\setlength{\cftbeforesubsecskip}{1pt}
\renewcommand{\cftsecfont}{\small\bfseries}
\renewcommand{\cftsubsecfont}{\small}
\renewcommand{\cftsecpagefont}{\small\bfseries}
\renewcommand{\cftsubsecpagefont}{\small}
\setlength{\headheight}{14pt}
\fancypagestyle{paperstyle}{
  \fancyhf{}
  \fancyhead[L]{\small\itshape Anclaje inerte y densidades p-adicas}
  \fancyhead[R]{\small\thepage}
  \renewcommand{\headrulewidth}{0.35pt}
  \renewcommand{\footrulewidth}{0pt}
}
\pagestyle{paperstyle}
\fancypagestyle{plain}{
  \fancyhf{}
  \fancyhead[R]{\small\thepage}
  \renewcommand{\headrulewidth}{0pt}
}

% ---------------------------------------------------------------
% Ajustes de maquetación profesional (floats + espaciados)
% ---------------------------------------------------------------
\setcounter{topnumber}{3}
\setcounter{bottomnumber}{2}
\setcounter{totalnumber}{5}
\renewcommand{\topfraction}{0.92}
\renewcommand{\bottomfraction}{0.80}
\renewcommand{\textfraction}{0.07}
\renewcommand{\floatpagefraction}{0.82}
\setlength{\textfloatsep}{10pt plus 2pt minus 2pt}
\setlength{\intextsep}{10pt plus 2pt minus 2pt}
\setlength{\abovecaptionskip}{6pt}
\setlength{\belowcaptionskip}{2pt}
\makeatletter
\def\fps@figure{htbp}
\def\fps@table{htbp}
\makeatother

% ---------------------------------------------------------------
% Entornos de teorema
% ---------------------------------------------------------------
\theoremstyle{plain}
\newtheorem{theorem}{Teorema}[section]
\newtheorem{proposition}[theorem]{Proposición}
\newtheorem{lemma}[theorem]{Lema}
\newtheorem{corollary}[theorem]{Corolario}
\newtheorem{conjecture}[theorem]{Conjetura}

\theoremstyle{definition}
\newtheorem{definition}[theorem]{Definición}
\newtheorem{example}[theorem]{Ejemplo}

\theoremstyle{remark}
\newtheorem{remark}[theorem]{Observación}
\numberwithin{equation}{section}
\allowdisplaybreaks[2]

% ---------------------------------------------------------------
% Notación y macros
% ---------------------------------------------------------------
\newcommand{\Z}{\mathbb{Z}}
\newcommand{\Q}{\mathbb{Q}}
\newcommand{\R}{\mathbb{R}}
\newcommand{\C}{\mathbb{C}}
\newcommand{\N}{\mathbb{N}}
\newcommand{\F}{\mathbb{F}}

\newcommand{\vp}{v_p}
\newcommand{\Zp}{\Z_p}
\newcommand{\euler}{\gamma}

\newcommand{\Csbase}{C_{\mathrm{base}}}
\newcommand{\Csinf}{C_{\infty}}
\newcommand{\Cobs}{C_{\sigma,\Delta}}
\newcommand{\Ssing}{\mathfrak{S}_\Delta}
\newcommand{\Sodd}{\mathfrak{S}_{\Delta,\mathrm{odd}}}

\newcommand{\MR}[2]{M_{#1}(#2)}
\newcommand{\Prob}{\mathbb{P}}
\newcommand{\E}{\mathbb{E}}

% ---------------------------------------------------------------
% Título (RECOMENDADO)
% ---------------------------------------------------------------
\title{\textbf{Anclaje Inerte y Densidades \(p\)-ádicas Explícitas en la familia \(R^2-c^2\):}\\[2pt]
\large Renormalización determinista y serie singular en campos cuadráticos imaginarios}
\author{Robert Duan Montoya Cardona}
\date{\today}
\setlength{\droptitle}{-1.2em}
\pretitle{\begin{center}\LARGE\bfseries}
\posttitle{\par\end{center}\vspace{0.45em}}
\preauthor{\begin{center}\normalsize}
\postauthor{\par\end{center}\vspace{0.2em}}
\predate{\begin{center}\small}
\postdate{\par\end{center}\vspace{0.65em}}

\begin{document}
\pagenumbering{roman}
\maketitle
\thispagestyle{empty}

% ================================================================
\begin{abstract}
Estudiamos la familia aritmética \(M_R(c)=R^2-c^2\), \(1\le c\le R-1\), asociada al conteo
\(N_\Delta(R)\) de valores que son normas en \(K=\Q(\sqrt{\Delta})\), con \(\Delta<0\) fundamental.
El resultado conceptual central es el \emph{anclaje inerte}: para un primo inerte \(p\), el defecto de paridad de
\(v_p(M_R(c))\) no es ruido estadístico, sino un fenómeno determinista concentrado en la capa crítica \(v_p(c)=v_p(R)\).
Esta rigidez produce fórmulas cerradas para densidades locales de norma:
\(\sigma_p^{(m)}=1-\frac{2}{p^m(p+1)}\) para \(p\ge 3\), y
\(\sigma_{2,\Delta}^{(m)}=1-\frac{1}{3\cdot 2^m}\) cuando \(2\) es inerte.

Con estas densidades definimos un normalizador finitista \(H_\Delta(R)\) que extrae de forma explícita la dependencia local en
la factorización de \(R\), y estabilizamos el observable
\(C_{\sigma,\Delta}(R)=\frac{N_\Delta(R)\log R}{H_\Delta(R)R}\).
Identificamos el baseline canónico
\(\Csbase(\Delta)=2e^{-\euler}L(1,\chi_\Delta)\), con forma cerrada en el caso Heegner, y proponemos la descomposición
\(\Csinf(\Delta)=\Csbase(\Delta)\Ssing\), donde \(\Ssing\) captura correlaciones locales finas.
La evidencia numérica hasta \(R\le 10^7\) para los nueve discriminantes de Heegner confirma estabilización robusta y estructura local no trivial.
Bajo una hipótesis estándar de nivel de distribución para \(M_R(c)\), obtenemos asintóticamente
\[
N_\Delta(R)=\Csinf(\Delta)\,H_\Delta(R)\,\frac{R}{\log R}
+O\!\left(\frac{R}{(\log R)^{1+\delta}}\right),\qquad \delta>0.
\]
\end{abstract}
\noindent\textbf{Palabras clave.} Normas cuadráticas imaginarias, densidades \(p\)-ádicas, cribas con serie singular, renormalización aritmética, anclaje inerte.
\par\noindent\textbf{MSC 2020.} 11N32, 11R11, 11N36, 11D09.

\clearpage
\tableofcontents
\clearpage
\pagenumbering{arabic}

% ================================================================
\section{Introducción y resumen de resultados}
\label{sec:intro}
% ================================================================

\subsection{Origen geométrico: de la esfera a la secuencia \(R^2-c^2\)}
El punto de partida es estrictamente geométrico:
\[
x^2+y^2+z^2=R^2.
\]
Al fijar una altura $z=c$, el paralelo cumple
\[
x^2+y^2=R^2-c^2.
\]
Por tanto, para cada $c$ aparece un escalar natural, $R^2-c^2$, que codifica el radio de la sección horizontal.
La pregunta aritmética nace al pedir que ese escalar pertenezca a un conjunto de normas.
La \Cref{fig:sphere-origin} resume este puente geométrico con una visualización directa.

\begin{figure}[htbp]
\centering
\begin{tikzpicture}[scale=0.95,>=Latex]
\pgfmathsetmacro{\Rs}{2.7}
\pgfmathsetmacro{\zc}{1.1}
\pgfmathsetmacro{\rc}{sqrt(\Rs*\Rs-\zc*\zc)}

\begin{scope}[xshift=-3.6cm]
  \draw[thick] (0,0) circle (\Rs);
  \draw[thick,->] (0,-3.1)--(0,3.3) node[above] {$z$};
  \draw[thick,->] (-3.1,0)--(3.2,0) node[right] {$x$};
  \draw[densely dashed,gray!70] (-\rc,\zc)--(\rc,\zc);
  \draw[blue!75!black,very thick] (-\rc,\zc)--(\rc,\zc);
  \fill[blue!70] (0,\zc) circle (1.4pt) node[left=2pt] {$c$};
  \draw[decorate,decoration={brace,amplitude=4pt}] (-\rc,\zc-0.25) -- (\rc,\zc-0.25)
    node[midway,below=6pt] {$2\sqrt{R^2-c^2}$};
  \node at (0,-3.55) {\small Corte meridional};
  \node[align=center] at (0,3.75) {$x^2+z^2=R^2$};
\end{scope}

\begin{scope}[xshift=4.2cm]
  \draw[thick] (0,0) circle (1.8);
  \draw[thick,->] (-2.2,0)--(2.4,0) node[right] {$x$};
  \draw[thick,->] (0,-2.2)--(0,2.3) node[above] {$y$};
  \draw[orange!85!black,very thick] (0,0) circle (1.8);
  \draw[<->,orange!85!black] (0,0)--(1.8,0)
    node[midway,above=2pt] {$\sqrt{R^2-c^2}$};
  \node[align=center] at (0,2.95) {$x^2+y^2=R^2-c^2$};
  \node at (0,-2.9) {\small Paralelo a altura $z=c$};
\end{scope}

\draw[->,line width=0.9pt] (-0.2,1.55) .. controls (0.9,2.3) and (2.2,2.1) .. (2.9,1.45);
\end{tikzpicture}
\caption{Origen geométrico de la familia. Cada corte horizontal de la esfera a altura $z=c$ produce un círculo de radio $\sqrt{R^2-c^2}$. El escalar $R^2-c^2$ es el objeto aritmético filtrado por normas.}
\label{fig:sphere-origin}
\end{figure}

\subsection{Por qué importa (lectura técnica y no técnica)}
\begin{itemize}[leftmargin=2em]
\item \textbf{Lectura no técnica.} El problema conecta geometría elemental (cortes de una esfera) con estructura fina de primos:
no es un promedio difuso, sino un patrón determinista gobernado por valuaciones.
\item \textbf{Lectura técnica.} La familia $M_R(c)=R^2-c^2$ permite escribir fórmulas locales exactas y aislar la dependencia en $v_p(R)$
mediante un normalizador explícito $H_\Delta(R)$; esto separa el bloque local del bloque global en la constante final.
\item \textbf{Impacto conceptual.} El caso gaussiano ($\Delta=-4$) es la puerta de entrada geométrica, y el formalismo de normas en
$K=\Q(\sqrt{\Delta})$ extiende esa puerta a toda una familia cuadrática con control local cerrado.
\end{itemize}

\subsection{Mapa lógico del artículo}
La \Cref{fig:roadmap-main} compacta la arquitectura completa: origen geométrico, filtros locales, renormalización y constante final.
\begin{figure}[htbp]
\centering
\begin{tikzpicture}[
  >=Latex,
  box/.style={
    draw=black!70,
    rounded corners=2.5pt,
    align=center,
    minimum height=1.2cm,
    text width=2.85cm,
    inner sep=4pt,
    font=\small
  }
]
\node[box,fill=cyan!8]   (a) at (-5.1,0) {Geometría\\$x^2+y^2+z^2=R^2$\\$z=c$};
\node[box,fill=cyan!8]   (b) at (-1.7,0) {Secuencia\\$n_c=R^2-c^2$};
\node[box,fill=orange!12](c) at (1.7,0)  {Filtro de norma\\$n_c\in N_{K/\Q}(\mathcal O_K)$};
\node[box,fill=orange!12](d) at (5.1,0)  {Densidades locales\\$\sigma_{p,\Delta}^{(m)}$};
\node[box,fill=green!10] (e) at (5.1,-2.0) {$H_\Delta(R)$\\(normalizador)};
\node[box,fill=green!10] (f) at (1.7,-2.0) {$C_{\sigma,\Delta}(R)$\\observable estable};
\node[box,fill=violet!10](g) at (-1.7,-2.0) {$\Csinf(\Delta)$\\$=\Csbase(\Delta)\Ssing$};

\draw[->,line width=0.85pt] (a)--(b);
\draw[->,line width=0.85pt] (b)--(c);
\draw[->,line width=0.85pt] (c)--(d);
\draw[->,line width=0.85pt] (d)--(e);
\draw[->,line width=0.85pt] (e)--(f);
\draw[->,line width=0.85pt] (f)--(g);
\end{tikzpicture}
\caption{Flujo conceptual completo: del origen geométrico de $R^2-c^2$ a la descomposición final de la constante global.}
\label{fig:roadmap-main}
\end{figure}

\subsection{El problema en una línea}
Sea $\Delta<0$ un discriminante fundamental y $K=\Q(\sqrt{\Delta})$.
Para cada $R\in\N$ consideramos la familia
\[
M_R(c)=R^2-c^2=(R-c)(R+c),\qquad 1\le c\le R-1,
\]
y contamos cuántos valores son normas enteras:
\[
N_\Delta(R)=\#\{\,1\le c\le R-1:\ M_R(c)\in N_{K/\Q}(\mathcal O_K)\,\}.
\]
El objetivo es describir con precisión \emph{cómo} depende $N_\Delta(R)$ de la aritmética de $R$ y del tipo de descomposición de primos en $K$.

\subsection{El fenómeno: oscilaciones locales deterministas (``anclaje inerte'')}
En un primo $p$ \emph{inerte} en $K$ (i.e.\ $\chi_\Delta(p)=-1$), una condición necesaria para que un entero positivo sea norma es que
$v_p(\,\cdot\,)$ sea \emph{par}. Lo sorprendente en la familia $M_R(c)$ es que, al variar $c$,
la paridad de $v_p(M_R(c))$ no se comporta como un ruido: queda rígidamente controlada por la relación aditiva
\[
(R-c)+(R+c)=2R.
\]
En particular, el defecto de paridad se concentra en una única capa $p$-ádica:

\medskip
\noindent\textbf{Principio de la capa crítica.}
\emph{Para $p$ impar, fuera del evento $v_p(c)=v_p(R)$ se tiene automáticamente que $v_p(M_R(c))$ es par.}
Por tanto, toda la probabilidad de fallo de paridad vive en un subconjunto de medida $\asymp p^{-v_p(R)}$.

\medskip
Este pegado del defecto a la valuación de $R$ (y no a fluctuaciones ``aleatorias'' de $c$) es lo que llamamos
\emph{anclaje inerte}. Es un mecanismo estructural, y por eso admite fórmulas cerradas.

\subsection{Resultados principales (locales, incondicionales)}
El núcleo del trabajo es local y completamente riguroso: calculamos densidades $p$-ádicas exactas asociadas a la paridad de $v_p(M_R(c))$
con $c$ Haar-uniforme en $\Z_p$.

\begin{itemize}[leftmargin=2em]
\item \textbf{Primos impares ($p\ge 3$).}
Para $m=v_p(R)$ obtenemos la identidad cerrada
\[
\Prob\big(v_p(M_R(c))\ \mathrm{impar}\big)=\frac{2}{p^m(p+1)},
\qquad
\sigma_p^{(m)}=\Prob\big(v_p(M_R(c))\ \mathrm{par}\big)=1-\frac{2}{p^m(p+1)}.
\]
En primos inertes, $\sigma_p^{(m)}$ es exactamente la densidad del filtro local de norma (Teorema~\ref{thm:local-odd}).

\item \textbf{Lugar $2$ cuando es inerte.}
Si $\Delta\equiv 5\pmod 8$, entonces $K_2/\Q_2$ es no ramificada y la condición local de norma es pura paridad.
En este régimen aparece un factor racional explícito:
\[
\sigma_{2,\Delta}^{(m)}=1-\frac{1}{3\cdot 2^{m}},\qquad \sigma_{2,\Delta}^{(0)}=\frac{2}{3}
\]
(Teorema~\ref{thm:sigma2-inerte}). Este \emph{interruptor $2$-ádico} domina varios de los regímenes observados numéricamente.
\end{itemize}

\subsection{El normalizador inercial \(H_\Delta(R)\) y el observable estable}
Las fórmulas anteriores muestran que las densidades dependen explícitamente de $m=v_p(R)$.
Por tanto, el cociente crudo $N_\Delta(R)\log R/R$ oscila multiplicativamente cuando $R$ acumula potencias de primos inertes.
Esto obliga a renormalizar por un corrector determinista construido desde las densidades locales.

Definimos el factor inercial finitista
\[
H_\Delta(R):=\prod_{\substack{p\mid R\\ p\ \mathrm{inerte}}}\frac{\sigma_{p,\Delta}^{(v_p(R))}}{\sigma_{p,\Delta}^{(0)}}
\]
y consideramos el observable
\[
C_{\sigma,\Delta}(R)=\frac{N_\Delta(R)\log R}{H_\Delta(R)\,R}.
\]
La razón conceptual para introducir $H_\Delta(R)$ es simple: extrae \emph{exactamente} la dependencia local explícita en $v_p(R)$,
devolviendo cada factor al régimen base $m=0$.

\subsection{Baseline canónico y serie singular}
En una criba de dimensión $1$ el escalamiento $R/\log R$ es el esperado, y el factor arquimediano queda forzado por Mertens.
En particular, identificamos el baseline canónico del campo:
\[
\Csbase(\Delta)=2e^{-\euler}L(1,\chi_\Delta)=2e^{-\euler}\operatorname{Res}_{s=1}\zeta_K(s),
\]
y en el caso Heegner ($h(\Delta)=1$) obtenemos la forma cerrada
\[
\Csbase(\Delta)=\frac{4\pi e^{-\euler}}{w(\Delta)\sqrt{|\Delta|}}
\]
(Teorema~\ref{thm:arch}).

La constante final esperada para el conteo renormalizado adopta la descomposición canónica
\[
\Csinf(\Delta)=\Csbase(\Delta)\,\mathfrak{S}_\Delta,
\]
donde $\mathfrak{S}_\Delta$ es una serie singular local que captura correlaciones finas (de orden $\ge p^{-2}$) propias de la familia
$M_R(c)$ y del conjunto de normas.

\subsection{Qué es teorema y qué es conjetura}
\begin{itemize}[leftmargin=2em]
\item \textbf{Teorema (incondicional).}
Todo el aparato local: principio de capa crítica, densidades cerradas para $p\ge 3$ y para $p=2$ inerte,
y la identificación del baseline $\Csbase(\Delta)$ vía $\zeta_K$.
\item \textbf{Teorema (global condicional).}
Bajo una hipótesis explícita de distribución en progresiones para $M_R(c)$, se deduce la asintótica global
con término de error con ahorro logarítmico (Teorema~\ref{thm:global-conditional}).
\item \textbf{Conjetura (global incondicional).}
El asintótico
\[
N_\Delta(R)\sim \frac{\Csinf(\Delta)\,H_\Delta(R)\,R}{\log R}
\qquad (R\to\infty)
\]
requiere un input global de distribución en progresiones para la secuencia $M_R(c)$, suficiente para alimentar una criba asintótica
(Conjetura~\ref{conj:global}).
\end{itemize}

\subsection{Evidencia computacional y el rol de inertes pequeños}
Para los nueve discriminantes de Heegner se computa $N_\Delta(R)$ hasta $R\le 10^7$ y se observa que $C_{\sigma,\Delta}(R)$ se estabiliza
hacia un límite $\Csinf(\Delta)$.
El caso $\Delta=-7$ actúa como diagnóstico: su valor atípico se explica casi por completo por el producto de primer orden de inertes pequeños
($p=3,5$), mostrando que la variación entre discriminantes es estructural y dominada por primos chicos (Sección~\ref{sec:ejemplo7}).

\subsection{Convención de parametrización}
\begin{remark}[Convención de escala y compatibilidad con la notación ``$2R$'']
\label{rem:convencion-escala}
En todo el manuscrito fijamos \emph{una única} parametrización:
\[
M_R(c)=R^2-c^2=(R-c)(R+c),\qquad 1\le c\le R-1.
\]
Si en algún contexto geométrico se prefiere escribir la familia como $(2r)^2-c^2$ (simetría alrededor de $2r$),
basta sustituir $R:=2r$. Para todo primo impar $p$ no cambia $v_p(R)$; en $p=2$ sí cambia:
\[
v_2(R)=v_2(2r)=v_2(r)+1.
\]
Esta observación es \emph{exactamente} el punto donde un cambio de convención puede alterar los ``casos base'' 2-ádicos.
La presente redacción fija la convención $R^2-c^2$ y usa $v_p(R)$ coherentemente en todo el texto.
\end{remark}

\subsection{Organización del manuscrito}
Las Secciones~\ref{sec:prelim-padic}--\ref{sec:local-2} prueban los resultados locales (impares y lugar $2$).
La Sección~\ref{sec:definiciones} formaliza $H_\Delta(R)$ y el observable $C_{\sigma,\Delta}(R)$.
Las Secciones~\ref{sec:global}--\ref{sec:renorm} discuten el baseline canónico, la renormalización tipo Mertens y la estructura Euleriana
de la serie singular. La evidencia numérica y los diagnósticos se presentan en la Sección~\ref{sec:computo} y apéndices.

% ================================================================
\section{Normas en $K=\Q(\sqrt{\Delta})$ y el rol de los primos inertes}
\label{sec:normas}
% ================================================================

\subsection{Caracteres cuadráticos y tipos de primos}
Sea $\Delta<0$ fundamental y $\chi_\Delta$ el carácter cuadrático de Kronecker. Para $p\nmid \Delta$,
\[
\chi_\Delta(p)=
\begin{cases}
+1 & (p\ \text{split}),\\
-1 & (p\ \text{inerte}).
\end{cases}
\]
Si $p\mid\Delta$, entonces $\chi_\Delta(p)=0$ y $p$ es ramificado.
Para contexto general sobre teoría analítica de números y aritmética de campos,
véanse \cite{Apostol,IwaniecKowalski,Neukirch}.

\subsection{Criterio local simplificado: paridad en primos inertes}
Un principio estándar (interpretado vía factorización de ideales) es:
\begin{quote}
\emph{Si $p$ es inerte, entonces para que $n$ sea norma es necesario que $\vp(n)$ sea par.}
\end{quote}
Esto se debe a que, si $p$ es inerte, el ideal $(p)$ permanece primo en $\mathcal{O}_K$ y
\[
(v_p(n)\ \text{impar})\ \Longrightarrow\ (p)\ \text{aparece con exponente impar en $(n)$},
\]
lo cual no puede suceder para normas principales, donde los exponentes vienen en pares conjugados.

Para los discriminantes de Heegner ($h(\Delta)=1$), esta descripción se vuelve especialmente operativa:
la representabilidad por la forma principal coincide con la condición global de norma principal, y la estructura local
captura con gran fidelidad la constante.

\begin{remark}[Por qué restringirse a Heegner en la evidencia]
Cuando $h(\Delta)>1$, un entero puede ser norma de algún ideal pero no necesariamente norma de un elemento, y aparecen clases.
La restricción a Heegner evita esas sutilezas y permite que el indicador de norma sea computable de manera directa
mediante criterios de factorización/representación.
\end{remark}

\begin{proposition}[Principio local--global para normas en extensiones cuadráticas]
\label{prop:hasse-norm}
Sea $K/\Q$ una extensión cuadrática (en particular, cíclica). Entonces un racional $x\in\Q^\times$ es una norma global
$x\in N_{K/\Q}(K^\times)$ si y solo si $x$ es una norma local en todo completamiento:
\[
x\in N_{K/\Q}(K^\times)\quad\Longleftrightarrow\quad
x\in N_{K_v/\Q_v}(K_v^\times)\ \ \text{para todo lugar $v$ de $\Q$}.
\]
\end{proposition}

\begin{remark}[Principio de Hasse no es suficiente: Normas Integrales]
\label{rem:hasse-vs-integral}
El Teorema de Normas de Hasse garantiza que $x\in\mathbb{Q}^\times$ es norma global en $K^\times$ si y solo si es norma local en todo completamiento $K_p^\times$.
Sin embargo, nuestro conteo $N_\Delta(R)$ requiere que $M_R(c)$ sea norma de un \emph{entero} algebraico ($\beta\in\mathcal{O}_K$).
Para $h(\Delta)=1$ (dominios de ideales principales), esto equivale a pedir que $M_R(c)$ sea generado por la norma de un elemento, lo cual impone condiciones más fuertes que la mera norma racional: se requiere que la valuación local sea par en inertes y que las unidades locales satisfagan congruencias cuadráticas en los ramificados (ver \Cref{subsec:ramificado}) y en $p=2$.
\end{remark}

% ================================================================
\section{Definiciones: conteo, densidades locales y el normalizador inercial}
\label{sec:definiciones}
% ================================================================

\subsection{Conteo}
\begin{definition}[Conteo de normas en la familia]
\label{def:N}
Para $\Delta<0$ fundamental definimos
\[
N_\Delta(R)=\#\{\,1\le c\le R-1:\ M_R(c)=R^2-c^2\in N_{K/\Q}(\mathcal{O}_K)\,\}.
\]
\end{definition}

\begin{remark}[Nota técnica sobre el borde $c=0$]
Formalmente, el valor $c=0$ produce $M_R(0)=R^2$, que es siempre norma (pues es un cuadrado racional).
Aunque su contribución es asintóticamente despreciable ($O(1)$ frente a $R$), incluir $c=0$ simplificaría ciertas sumas de caracteres en un tratamiento analítico puro.
Mantenemos la restricción $c\ge 1$ por coherencia con la intuición geométrica del problema.
\end{remark}

\subsection{Densidades $p$-ádicas}
El mecanismo central del artículo es local. Sea $p$ primo. Consideramos $c$ uniforme en $\Z_p$ con medida de Haar.
Definimos la densidad local (en el régimen relevante para inertes):
\[
\sigma_{p,\Delta}^{(m)}:=\Prob_{c\in\Z_p}\big(\vp(M_R(c))\ \text{par}\big),
\qquad m=\vp(R).
\]
Cuando $p$ es inerte, esta paridad modela la condición local de norma.

\subsection{Por qué $H_\Delta(R)$ es el normalizador natural}
El punto crucial es que $\sigma_{p,\Delta}^{(m)}$ depende de $m=\vp(R)$ de manera explícita.
Por tanto, si $R$ tiene potencias grandes de primos inertes, el conteo $N_\Delta(R)$ cambia por un factor determinista.
Esto obliga a definir un corrector multiplicativo.

\begin{definition}[Factor inercial determinista finitista]
\label{def:H}
Sea $\Delta<0$ fundamental. Definimos el factor corrector como el producto finito restringido a los primos relevantes:
\[
H_\Delta(R):=\prod_{\substack{p\mid R\\ p\ \mathrm{inerte}}} \frac{\sigma_{p,\Delta}^{(m)}}{\sigma_{p,\Delta}^{(0)}},
\qquad m=v_p(R).
\]
Si $2$ es inerte, el producto incluye también el factor $p=2$ cuando $v_2(R)>0$.
Para primos split, y para inertes con $v_p(R)=0$, el cociente es $1$ (o se absorbe en la constante global para correcciones de orden $O(p^{-2})$).
Esta definición operacional garantiza que $H_\Delta(R)$ es una cantidad finita y computable que aísla la dependencia en la geometría de $R$. Los términos no triviales provienen de:
\begin{enumerate}
    \item Primos \textbf{inertes} ($p\mid R, \chi(p)=-1$), donde la paridad se ve alterada por $m>0$.
    \item El primo $p=2$ cuando es \textbf{inerte} y divide a $R$.
\end{enumerate}
\end{definition}

\begin{remark}[Lectura probabilística]
Si los eventos locales fueran aproximadamente independientes, entonces
\[
\Prob(M_R(c)\ \text{pasa todos los filtros locales})\ \approx\ \prod_{p} \sigma_{p,\Delta}^{(m)},
\]
y al variar $v_p(R)$ cambias exactamente las $\sigma_{p,\Delta}^{(m)}$ correspondientes. $H_\Delta(R)$ es
el cociente que ``devuelve'' esos factores a su valor base $m=0$.
\end{remark}

\begin{remark}[Refinamiento ramificado opcional]
Para estudiar explícitamente la contribución de primos impares ramificados, usamos la normalización post-procesada
\[
H_\Delta^\star(R):=H_\Delta(R)\prod_{\substack{p\mid \Delta\\ p\ \text{impar}}}p^{-v_p(R)}.
\]
Este refinamiento no altera la definición operativa principal de los CSV (inerte-only), pero permite separar con claridad el bloque inerte del bloque ramificado.
\end{remark}

\subsection{El observable estable}
\begin{definition}[Observable normalizado]
\label{def:Cobs}
Definimos
\[
C_{\sigma,\Delta}(R):=\frac{N_\Delta(R)\log R}{H_\Delta(R)\,R}.
\]
\end{definition}

\begin{remark}[Qué esperamos en un modelo de dimensión 1]
En una criba de dimensión $1$, el conteo típico tiene forma $\sim \mathrm{const}\cdot R/\log R$.
La función $H_\Delta(R)$ corrige las oscilaciones locales y $C_{\sigma,\Delta}(R)$ debería tender a un límite.
\end{remark}

\begin{remark}[Convención de escala logarítmica]
En un modelo de criba de dimensión $1$ aplicado a valores de tamaño $\asymp R^2$, el denominador natural es $\log(R^2)=2\log R$.
Al normalizar por $\log R$, absorbemos ese factor $2$ en la constante $\Csbase(\Delta)$, justificando su prefactor.
\end{remark}

\begin{figure}[htbp]
\centering
\incfig[width=\linewidth]{../../Figuras/Articulo/D-43_raw_vs_norm.pdf}
\caption{Demostración visual del anclaje inerte para $\Delta=-43$. \textbf{Arriba:} $C_{\rm raw}(R)=N_\Delta(R)\log R/R$ oscila multiplicativamente. \textbf{Abajo:} al normalizar por $H_\Delta(R)$, el observable $C_{\sigma,\Delta}(R)$ se estabiliza, evidenciando que $H_\Delta$ captura la oscilación determinista inducida por primos inertes.}
\label{fig:D43-raw-vs-norm}
\end{figure}

% ================================================================
\section{Preliminares $p$-ádicos: capas de valuación y el principio de la capa crítica}
\label{sec:prelim-padic}
% ================================================================

\subsection{Puente: de lo discreto a lo continuo}
\begin{lemma}[Densidad discreta vs Haar]
\label{lem:haar-bridge}
Sea $\sigma_{p}^{(m)}$ la probabilidad respecto a la medida de Haar en $\Z_p$ definida previamente (densidad intrínseca).
Definimos la \emph{densidad discreta} para el nivel $R$ como:
\[
\sigma_{p,\text{disc}}^{(m)}(R):=\frac{1}{R-1}\#\{\,1\le c\le R-1:\ v_p(M_R(c))\ \text{es par}\,\}.
\]
Entonces, para todo $q=p^k$ fijo (o creciendo lentamente con $R$), la equidistribución de $c$ módulo $q$ implica:
\[
\sigma_{p,\text{disc}}^{(m)}(R)=\sigma_{p}^{(m)}+O\Big(\frac{p^{m+1}}{R}\Big).
\]
\end{lemma}
\begin{remark}
La corrección inercial $H_\Delta(R)$ se construye usando las densidades límite $\sigma_{p}^{(m)}$.
Esta aproximación es válida y robusta siempre que las ``capas'' relevantes del espacio $p$-ádico (hasta valuación $m$) tengan representantes en el intervalo discreto, lo cual se cumple ampliamente si $p^m \ll R$.
Para $R$ extremadamente $p$-suave (ej. $R=p^m$), la discretización introduce efectos de borde finito que el modelo de Haar suaviza.
\end{remark}

\subsection{Capas de valuación en $\Z_p$}
Usaremos repetidamente la descomposición de $\Z_p$ por valuaciones.

\begin{lemma}[Distribución de $\vp$ bajo Haar]
\label{lem:geom-vp}
Sea $X$ uniforme en $\Z_p$. Entonces para $k\ge 0$,
\[
\Prob(\vp(X)=k)=(1-1/p)p^{-k},
\qquad
\Prob(\vp(X)\ge k)=p^{-k}.
\]
\end{lemma}
\begin{proof}
$\Prob(\vp(X)\ge k)=\mu(p^k\Z_p)=p^{-k}$. Restando $k$ y $k+1$ se obtiene la masa puntual.
\end{proof}

\subsection{El principio estructural: todo el defecto vive en $\vp(c)=\vp(R)$}
El polinomio
\[
M_R(c)=(R-c)(R+c)
\]
tiene la propiedad de que las valuaciones $\vp(R\pm c)$ se estabilizan fuera de la capa crítica.

\begin{lemma}[Fuera de la capa crítica la paridad es automáticamente par]
\label{lem:structure-vp}
Sea $p$ primo impar. Escriba $R=p^mR_0$ con $p\nmid R_0$ y sea $c\in\Z_p$ con $t=\vp(c)$.
Si $t\neq m$, entonces
\[
\vp(R-c)=\vp(R+c)=\min\{t,m\},
\quad\Rightarrow\quad
\vp(M_R(c))=2\min\{t,m\}\ \text{es par}.
\]
\end{lemma}
\begin{proof}
Si $t<m$, escriba $c=p^tu$ con $u\in\Z_p^\times$ y note que
$R\pm c=p^t(p^{m-t}R_0\pm u)$ con paréntesis unidad. Si $t>m$, escriba
$c=p^m p^{t-m}u$ con $p^{t-m}u\in p\Z_p$ y observe que $R\pm c=p^m(R_0\pm p^{t-m}u)$ con paréntesis unidad.
\end{proof}

\begin{lemma}[Reducción en la capa crítica]
\label{lem:critical-reduction}
Sea $p$ impar. Si $\vp(c)=m$ escribimos $c=p^m u$ con $u\in\Z_p^\times$. Entonces
\[
\vp(M_R(c))=2m+\vp\big(R_0^2-u^2\big).
\]
En particular, la paridad de $\vp(M_R(c))$ coincide con la de $\vp(R_0^2-u^2)$.
\end{lemma}
\begin{proof}
Factorizando $M_R(c)=R^2-c^2=p^{2m}(R_0^2-u^2)$.
\end{proof}

\begin{remark}[Consecuencia conceptual]
Esta es la razón de fondo por la que $H_\Delta(R)$ existe y es simple: la dependencia de $\sigma_{p,\Delta}^{(m)}$ respecto de $\vp(R)$
se concentra en una capa de masa $\asymp p^{-m}$.
\end{remark}
La \Cref{fig:critical-layer} lo visualiza en un solo diagrama: todo el defecto de paridad entra por $t=\vp(c)=m$.

\begin{figure}[htbp]
\centering
\begin{tikzpicture}[
  >=Latex,
  zone/.style={draw=black!70,rounded corners=2pt,minimum width=3.7cm,minimum height=1.15cm,align=center,font=\small}
]
\node[zone,fill=green!12]  (left)   at (0,0) {$t=\vp(c)<m$\\$\vp(M_R(c))=2t$\\\textbf{par siempre}};
\node[zone,fill=orange!18] (middle) at (4.7,0) {$t=\vp(c)=m$\\\textbf{capa crítica}\\puede aparecer impar};
\node[zone,fill=green!12]  (right)  at (9.4,0) {$t=\vp(c)>m$\\$\vp(M_R(c))=2m$\\\textbf{par siempre}};

\draw[->,line width=0.9pt] (left)--(middle);
\draw[->,line width=0.9pt] (middle)--(right);

\draw[decorate,decoration={brace,amplitude=4pt},gray!70]
  (3.0,1.0) -- (6.4,1.0)
  node[midway,above=5pt,font=\small] {masa $\asymp p^{-m}$};

\node[draw=black!70,rounded corners=2pt,fill=white,align=center,font=\small,inner sep=3pt] at (4.7,-1.35)
{Defecto de paridad concentrado\\en una sola capa de valuación.};
\end{tikzpicture}
\caption{Visualización del principio de capa crítica para $p$ impar: fuera de $t=\vp(c)=m$ la paridad de $\vp(M_R(c))$ es forzosamente par.}
\label{fig:critical-layer}
\end{figure}

% ================================================================
\section{Teoremas locales en primos impares ($p\ge 3$)}
\label{sec:local-impar}
% ================================================================

\subsection{Caso Inerte: Paridad de Valuación}
El siguiente hecho encapsula el conteo de soluciones $p$-ádicas cerca de raíces simples.

\begin{lemma}[Conteo de levantamientos para raíces simples en unidades]
\label{lem:hensel-units}
Sea $p$ primo impar y $h(T)\in\Z_p[T]$. Suponga que $\bar h(T)\in(\Z/p\Z)[T]$ tiene exactamente $r$ raíces simples en
$(\Z/p\Z)^\times$. Sea $U$ uniforme en $\Z_p^\times$.
Entonces para $t\ge 1$:
\[
\Prob\big(h(U)\equiv 0\!\!\pmod{p^t}\big)=\frac{r}{\varphi(p^t)}=\frac{r}{(p-1)p^{t-1}},
\qquad
\Prob\big(\vp(h(U))=t\big)=\frac{r}{p^t}.
\]
\end{lemma}
\begin{proof}
Cada raíz simple módulo $p$ levanta de manera única a una raíz módulo $p^t$ (Hensel).
Por tanto hay exactamente $r$ soluciones módulo $p^t$ dentro de $(\Z/p^t\Z)^\times$.
Como $U\bmod p^t$ es uniforme en ese grupo (de tamaño $\varphi(p^t)=(p-1)p^{t-1}$), obtenemos la primera fórmula.
Para la segunda, restamos probabilidades:
\[
\Prob(\vp(h(U))=t)=\Prob(\vp(h(U))\ge t)-\Prob(\vp(h(U))\ge t+1)
=\frac{r}{(p-1)p^{t-1}}-\frac{r}{(p-1)p^{t}}=\frac{r}{p^t}.
\]
\end{proof}

\begin{remark}
En \Cref{lem:hensel-units} sólo usamos $t\ge 1$ porque la paridad de $\vp(h(U))$ difiere de $0$
únicamente cuando $p\mid h(U)$.
\end{remark}

\subsection{Densidad local cerrada}
\begin{theorem}[Densidad local para $p\ge 3$]
\label{thm:local-odd}
Sea $p\ge 3$ primo impar y $R=p^mR_0$ con $p\nmid R_0$. Entonces
\[
\boxed{
\Prob_{c\in\Z_p}\big(\vp(M_R(c))\ \mathrm{impar}\big)=\frac{2}{p^m(p+1)}
}
\]
\[
\boxed{
\sigma_{p}^{(m)}=\Prob(\vp(M_R(c))\ \mathrm{par})=1-\frac{2}{p^m(p+1)}.
}
\]
En particular, para $m=0$:
\[
\boxed{\sigma_{p}^{(0)}=\frac{p-1}{p+1}.}
\]
\noindent (Cuando $p$ es inerte, esta densidad es exactamente el filtro local de norma).
\end{theorem}

\begin{proof}
Por \Cref{lem:structure-vp}, fuera de la capa crítica $\vp(c)=m$ la valuación $\vp(M_R(c))$ es par.
Así,
\[
\Prob(\vp(M_R(c))\ \mathrm{impar})
=\Prob(\vp(c)=m)\cdot \Prob(\vp(M_R(c))\ \mathrm{impar}\mid \vp(c)=m).
\]
Por \Cref{lem:geom-vp}, $\Prob(\vp(c)=m)=(1-1/p)p^{-m}=(p-1)/p^{m+1}$.

Condicionando a $\vp(c)=m$ escribimos $c=p^mu$ con $u\in\Z_p^\times$ uniforme.
Por \Cref{lem:critical-reduction},
\[
\vp(M_R(c))=2m+\vp\big(R_0^2-u^2\big),
\]
luego la paridad es la de $\vp(h(u))$ con $h(T)=R_0^2-T^2$.

Módulo $p$, $\bar h(T)=\bar A^2-T^2$ con $\bar A\neq 0$ tiene exactamente $r=2$ raíces simples en $(\Z/p\Z)^\times$:
$T\equiv \pm \bar A$. Por \Cref{lem:hensel-units}, para $t\ge 1$ (el caso $t=0$ no contribuye a sumas de valuación):
\[
\Prob(\vp(h(u))=t)=\frac{2}{p^t}.
\]
Por tanto,
\[
\Prob(\vp(h(u))\ \mathrm{impar})=\sum_{j\ge 0}\Prob(\vp(h(u))=2j+1)
=\sum_{j\ge 0}\frac{2}{p^{2j+1}}
=\frac{2/p}{1-1/p^2}
=\frac{2p}{p^2-1}.
\]
Finalmente,
\[
\Prob(\vp(M_R(c))\ \mathrm{impar})
=\frac{p-1}{p^{m+1}}\cdot \frac{2p}{p^2-1}
=\frac{2}{p^m(p+1)}.
\]
Esto concluye.
\end{proof}

\begin{remark}[Lectura intuitiva]
En $m=0$, la probabilidad de fallo en un inerte impar es $2/(p+1)$ y por tanto la probabilidad de pasar el filtro es $(p-1)/(p+1)$.
Esto explica por qué unos pocos inertes pequeños pueden dominar la constante (p.ej. $p=3,5$ para $\Delta=-7$).
\end{remark}

\subsection{Primos ramificados impares en Heegner: caso explícito ($\Delta=-3$, $p=3$)}
\label{subsec:ramificado}

En la lista de discriminantes de Heegner $h(\Delta)=1$, cada primo impar $p\mid\Delta$ puede afectar el conteo cuando $v_p(R)>0$.
El caso que desarrollamos de forma completamente explícita en el cuerpo del texto es $(\Delta,p)=(-3,3)$.
El análisis completo (incluyendo la caracterización del grupo de normas de unidades módulo $3$ y el conteo en la capa crítica)
se presenta en el Apéndice, \Cref{subsec:sigma3-minus3}, donde se prueba la fórmula cerrada
\[
\sigma_{3,-3}^{(m)}=\frac{2}{3^{m+1}}.
\]
Para otros Heegner impares ($\Delta=-7,-11,-19,-43,-67,-163$), el primo $|\Delta|$ también induce oscilación visible al variar $v_p(R)$;
esto se refleja en el refinamiento $H_\Delta^\star$ de la Sección~\ref{sec:computo}.

% ================================================================
\section{Capítulo 2-ádico: el interruptor racional en el caso inerte}
\label{sec:local-2}
% ================================================================

\subsection{Clasificación de $2$ (incluida para no depender de folklore)}
\begin{lemma}[Clasificación de $2$ por congruencia módulo $8$]
\label{lem:clasif-2}
Sea $\Delta<0$ discriminante fundamental.
\begin{enumerate}
    \item Si $\Delta$ es \textbf{par} fundamental (ej. $\Delta=-4,-8$), entonces $2$ es \textbf{ramificado}.
    \item Si $\Delta$ es \textbf{impar} fundamental, entonces $\Delta\equiv 1\pmod 4$.
    \begin{itemize}
        \item $\Delta\equiv 1\pmod 8 \implies 2$ \textbf{split}.
        \item $\Delta\equiv 5\pmod 8 \implies 2$ \textbf{inerte}.
    \end{itemize}
\end{enumerate}
\end{lemma}

\subsection{Un lema 2-ádico auxiliar (para que la prueba no ``salte'')}
\begin{lemma}[Suma y diferencia de impares en $\Z_2$]
\label{lem:odd-sum-diff}
Si $a,b\in\Z_2$ son impares, entonces $a\pm b$ son pares y exactamente uno de ellos tiene valuación $1$.
Más precisamente, uno de $a\pm b$ es $\equiv 2\pmod 4$ y el otro es $\equiv 0\pmod 4$.
\end{lemma}
\begin{proof}
Si $a\equiv b\pmod 4$, entonces $a-b\equiv 0\pmod 4$ y $a+b\equiv 2\pmod 4$.
Si $a\equiv -b\pmod 4$, se invierte el rol. En cualquier caso, ambos son pares y sólo uno es múltiplo de $4$.
\end{proof}

\subsection{Teorema 2-ádico en el caso inerte}
Antes de probar el teorema principal, aislamos una propiedad geométrica elemental de la medida de Haar que se usa implícitamente al ``contar potencias de 2 adicionales''.

\begin{lemma}[Valuación geométrica condicionada]
\label{lem:geom-shift}
Si $X$ es una variable aleatoria uniforme en $\Z_2$ (Haar), entonces la valuación de $X$ condicionada a estar en un subgrupo $2^k\Z_2$ sigue una ley geométrica desplazada:
\[
\Prob(v_2(X)=j \mid X\in 2^k\Z_2) = \frac{1}{2^{j-k+1}},\qquad \forall j\ge k.
\]
En particular, si sabemos que $X\in 4\Z_2$, entonces $\Prob(v_2(X)=j)=2^{-(j-1)}$ para $j\ge 2$.
\end{lemma}

Para el lugar $2$ usamos la misma notación $M_R(c)$ (o $F_R(c)$ si se quiere enfatizar la variable muda) consistente con el resto del texto.

\begin{theorem}[Densidad local 2-ádica en el caso inerte]
\label{thm:sigma2-inerte}
Supongamos que $2$ es inerte en $K$ (impar fundamental con $\Delta\equiv 5\pmod 8$).
Sea $m=v_2(R)$ y sea $c$ uniforme en $\Z_2$ (Haar). Entonces
\[
\sigma_{2,\Delta}^{(m)}:=\Prob\big(v_2(M_R(c))\ \mathrm{par}\big)
\]
satisface la fórmula cerrada
\[
\boxed{\;\sigma_{2,\Delta}^{(m)}=1-\frac{1}{3\cdot 2^{m}}.\;}
\]
En particular, para $R$ impar ($m=0$),
\[
\boxed{\;\sigma_{2,\Delta}^{(0)}=\frac{2}{3}.\;}
\]
\end{theorem}

\begin{corollary}[Fórmula cerrada para $H_\Delta(R)$]
\label{cor:H-formula}
El factor inercial toma la forma explícita:
\[
H_\Delta(R)=\prod_{\substack{p\mid R \\ p\ \mathrm{inerte}}} h_p(v_p(R)),
\]
donde el multiplicador local $h_p(m)=\sigma_{p,\Delta}^{(m)}/\sigma_{p,\Delta}^{(0)}$ es:
\begin{itemize}
    \item Para $p\ge 3$ inerte:
    \[
    h_p(m)=\frac{p^m(p+1)-2}{p^m(p-1)}.
    \]
    \item Para $p=2$ inerte (si aplica):
    \[
    h_2(m)=\frac{3\cdot 2^m - 1}{2^{m+1}}.
    \]
\end{itemize}
\end{corollary}

\begin{proof}
Escribimos $R=2^m r$ con $r$ impar y descomponemos por capas $t=v_2(c)$.

\medskip
\noindent\textbf{Paso 1: fuera de la capa crítica $t\neq m$ no hay defecto.}
Si $t<m$, escribimos $c=2^t c'$ con $c'$ impar. Entonces
\[
R\pm c=2^t(2^{m-t}r\pm c'),
\]
y como $2^{m-t}r$ es par y $c'$ impar, el paréntesis es impar, luego $v_2(R\pm c)=t$.
Si $t>m$, escribimos $c=2^m\cdot 2^{t-m}c'$ con $2^{t-m}c'$ par y
\[
R\pm c=2^m(r\pm 2^{t-m}c'),
\]
con paréntesis impar; por tanto $v_2(R\pm c)=m$.
En ambos casos
\[
v_2(M_R(c))=v_2(R-c)+v_2(R+c)=2\min\{t,m\},
\]
que es par.

\medskip
\noindent\textbf{Paso 2: masa de la capa crítica.}
La probabilidad de $v_2(c)=m$ en $\Z_2$ es
\[
\Prob(v_2(c)=m)=2^{-(m+1)}.
\]

\medskip
\noindent\textbf{Paso 3: defecto condicionado a $v_2(c)=m$.}
Condicionamos a $v_2(c)=m$ y escribimos $c=2^m c'$ con $c'$ impar uniforme entre impares 2-ádicos. Entonces
\[
v_2(M_R(c))=2m+v_2(r-c')+v_2(r+c').
\]
Por \Cref{lem:odd-sum-diff}, como $r$ y $c'$ son impares, exactamente uno de $r\pm c'$ tiene valuación $1$ y el otro tiene valuación $\ge 2$.
Luego existe $k\ge 2$ tal que
\[
\{v_2(r-c'),v_2(r+c')\}=\{1,k\},
\quad\Rightarrow\quad
v_2(r-c')+v_2(r+c')=1+k.
\]
Así, la paridad de $v_2(M_R(c))$ es par si y sólo si $k$ es impar.

Nos queda calcular $\Prob(k\ \text{impar})$. Partimos en dos ramas equiprobables según clases módulo $4$:
\[
\mathcal A_+:=\{c'\equiv r\!\!\!\pmod 4\},\qquad
\mathcal A_-:=\{c'\equiv -r\!\!\!\pmod 4\},
\]
cada una de masa $1/2$ dentro de las unidades impares.

En $\mathcal A_+$ se tiene $v_2(r-c')\ge 2$ y $v_2(r+c')=1$. Definiendo $X_+=(r-c')/2$, resulta $X_+\in 2\Z_2$ y
\[
k=v_2(r-c')=1+v_2(X_+).
\]
Por \Cref{lem:geom-shift}, condicionado a $X_+\in 2\Z_2$:
\[
\Prob(v_2(X_+)=j)=2^{-j}\qquad (j\ge 1).
\]
Entonces
\[
\Prob(k\ \text{impar}\mid \mathcal A_+)
=\Prob(1+v_2(X_+)\ \text{impar})
=\sum_{\substack{j\ge 1\\ j\ \mathrm{par}}}2^{-j}
=\frac{1}{3}.
\]

En $\mathcal A_-$ se intercambian los papeles de $r-c'$ y $r+c'$. Definiendo $X_-=(r+c')/2\in 2\Z_2$, el mismo argumento da
\[
\Prob(k\ \text{impar}\mid \mathcal A_-)=\frac{1}{3}.
\]
Por la ley de probabilidad total:
\[
\Prob(k\ \text{impar})
=\frac12\cdot\frac13+\frac12\cdot\frac13
=\frac13.
\]
Luego, condicionado a $v_2(c)=m$, el \emph{éxito} (paridad par) ocurre con probabilidad $1/3$
y el \emph{fallo} con probabilidad $2/3$.

\medskip
\noindent\textbf{Paso 4: combinar.}
\[
\Prob(\text{fallo})=\Prob(v_2(c)=m)\cdot \Prob(\text{fallo}\mid v_2(c)=m)
=2^{-(m+1)}\cdot \frac{2}{3}=\frac{1}{3\cdot 2^m}.
\]
Por complemento,
\[
\sigma_{2,\Delta}^{(m)}=1-\frac{1}{3\cdot 2^m}.
\]
\end{proof}

\begin{corollary}[Contraste: caso gaussiano $\Delta=-4$ (2 ramificado)]
\label{cor:gauss2}
En $K=\Q(i)$ el primo $2$ es ramificado. La condición clásica para ser suma de dos cuadrados
impone paridad en exponentes de primos $p\equiv 3\pmod 4$, pero no introduce una obstrucción específica en $v_2$ del tipo “paridad forzada”.
En el lenguaje de este trabajo: $2$ no pertenece al conjunto de primos inertes y, por tanto, no contribuye un factor del tipo
$\sigma_{2,\Delta}^{(m)}$ al filtro inerte. En particular, el ``interruptor'' 2-ádico que aparece cuando $2$ es inerte está ausente en $\Delta=-4$.
\end{corollary}

\begin{remark}[Lectura del factor racional en el caso $2$-inerte]
En discriminantes con $2$ inerte, el Teorema~\ref{thm:sigma2-inerte} fija un factor racional explícito en función de $m=v_2(R)$.
En particular, para $R$ impar es exactamente $2/3$.
Cuando se adopta la convención alternativa $R=2r$ (ver \Cref{rem:convencion-escala}), el caso base cambia porque $v_2(R)=v_2(2r)=v_2(r)+1$.
\end{remark}

% ================================================================
\section{Modelo global y baseline canónico: por qué aparece $\zeta$ y por qué aparece $e^{-\euler}$}
\label{sec:global}
% ================================================================

\subsection{La conjetura global en forma canónica}
\begin{conjecture}[Ley global (criba de dimensión 1 con factor inercial)]
\label{conj:global}
Para cada $\Delta<0$ fundamental existe una constante $\Csinf(\Delta)>0$ tal que
\[
\boxed{
N_\Delta(R)\ \sim\ \frac{\Csinf(\Delta)\,H_\Delta(R)\,R}{\log R}
\qquad (R\to\infty).
}
\]
Equivalente:
\[
\boxed{
C_{\sigma,\Delta}(R)=\frac{N_\Delta(R)\log R}{H_\Delta(R)\,R}\ \longrightarrow\ \Csinf(\Delta).
}
\]
\end{conjecture}

\begin{remark}[Input necesario: nivel de distribución]
Para convertir esta conjetura en teorema (vía criba asintótica), no basta con las densidades locales.
Se requiere probar un resultado de ``nivel de distribución'' para la secuencia $M_R(c)=R^2-c^2$: demostrar que el término de error en el conteo de soluciones módulo $d$ es controlable en promedio para $d$ hasta $R^\theta$ con algún $\theta>0$ (idealmente $\theta$ grande).
Sin este input global analítico, las densidades $p$-ádicas sólo describen la estructura local, no la distribución asintótica del conteo.
\end{remark}

\subsection{La identidad madre: $\zeta_K=\zeta\,L$}
El campo cuadrático $K$ tiene zeta de Dedekind $\zeta_K(s)$. Para extensiones cuadráticas:
\[
\boxed{\zeta_K(s)=\zeta(s)\,L(s,\chi_\Delta).}
\]
Esta identidad codifica splitting primo a primo y explica por qué el término natural del campo es $L(1,\chi_\Delta)$:
como $\zeta(s)\sim 1/(s-1)$ y $L(s,\chi_\Delta)$ es holomorfa y no nula en $s=1$ para $\Delta<0$,
\[
\boxed{\operatorname{Res}_{s=1}\zeta_K(s)=L(1,\chi_\Delta).}
\]

\subsection{El baseline arquimediano y la fórmula de clase}
\begin{theorem}[Baseline canónico: residuo de $\zeta_K$ y fórmula de clase]
\label{thm:arch}
Sea $\Delta<0$ fundamental y $K=\Q(\sqrt{\Delta})$. Definimos
\[
\boxed{\Csbase(\Delta):=2e^{-\euler}L(1,\chi_\Delta)=2e^{-\euler}\operatorname{Res}_{s=1}\zeta_K(s).}
\]
Si $\Delta$ es de Heegner ($h(\Delta)=1$), entonces
\[
\boxed{\Csbase(\Delta)=\frac{4\pi e^{-\euler}}{w(\Delta)\sqrt{|\Delta|}}.}
\]
\end{theorem}
\begin{proof}
La primera igualdad se sigue de $\operatorname{Res}_{s=1}\zeta_K(s)=L(1,\chi_\Delta)$.
La fórmula analítica del número de clases para campos cuadráticos imaginarios da
\[
L(1,\chi_\Delta)=\frac{2\pi\,h(\Delta)}{w(\Delta)\sqrt{|\Delta|}}.
\]
Para Heegner, $h(\Delta)=1$.
\end{proof}

\subsection{Por qué aparece $e^{-\euler}$ (y por qué esto es inevitable)}
La constante $e^{-\euler}$ aparece como renormalización natural de la divergencia armónica
$\sum_{p\le x}\frac{1}{p}$: el teorema de Mertens implica
\[
\prod_{p\le x}\Big(1-\frac{1}{p}\Big)\sim \frac{e^{-\euler}}{\log x}.
\]
Toda criba de dimensión $1$ produce un factor $e^{-\euler}$ al convertir ``producto sobre primos'' en ``escala $1/\log$''.
En nuestro caso, la parte del campo se canaliza por $L(1,\chi_\Delta)$; el resto pertenece a la geometría aritmética de la familia $M_R(c)$.

\subsection{Descomposición canónica}
La forma natural de la constante final es
\[
\boxed{
\Csinf(\Delta)
=
\underbrace{2e^{-\euler}L(1,\chi_\Delta)}_{\Csbase(\Delta)}
\times
\underbrace{\kappa_2(\Delta)\prod_{p\ge 3}\kappa_p(\Delta)}_{\Ssing}.
}
\]
Equivalentemente,
\[
\Ssing=\kappa_2(\Delta)\,\Sodd,
\qquad
\Sodd:=\prod_{p\ge 3}\kappa_p(\Delta).
\]
Formalmente, los factores $\kappa_p(\Delta)$ se definen como límites de densidades locales normalizadas
por el cociente de primer orden
\[
(1-1/p)\bigl(1-\chi_\Delta(p)/p\bigr)^{-1},
\]
de modo que la contribución residual empieza en orden $p^{-2}$:
\[
\kappa_p(\Delta) = \lim_{k\to\infty} \frac{\Prob_{c\in \Z/p^k\Z}(M_R(c)\in N(\mathcal{O}_{K,p}))}{(1-1/p)(1-\chi_\Delta(p)/p)^{-1}}.
\]
Este límite estabiliza en cuanto $p^k$ supera la ramificación y la capa crítica (que es $m=0$ para $\Ssing$).
En \Cref{sec:renorm} mostramos que, con esta normalización, el producto infinito queda absolutamente convergente.
Si definimos la parte impar
\[
\Sodd := \prod_{p\ge 3}\kappa_p(\Delta),
\]
entonces $\Ssing=\kappa_2(\Delta)\Sodd$ (cuando $\Delta$ tiene $2$ inerte o ramificado).
donde $\kappa_p(\Delta)$ es un factor local (serie singular) capturando la interacción específica del evento
``$M_R(c)$ es norma'' con la estructura $p$-ádica del polinomio y con el tipo del primo (inerte/split/ramificado).

% ================================================================
\section{Estructura Euleriana: aparición forzada de $\zeta$, $L$ y expansión a orden $p^{-2}$}
\label{sec:zetaL}
% ================================================================

\subsection{Factores locales $\kappa_p(\Delta)$: definición operacional}
En este trabajo distinguimos dos niveles:

\begin{itemize}[leftmargin=2em]
\item \emph{Filtro inerte de primer orden:} para primos inertes (incluido $p=2$ cuando es inerte), la condición local
``$M_R(c)$ es norma'' reduce a una restricción explícita sobre $v_p(M_R(c))$ (paridad), cuya densidad es $\sigma_{p,\Delta}^{(m)}$.

\item \emph{Serie singular fina:} incluso si todas las densidades marginales $\sigma_{p,\Delta}^{(0)}$ están dadas,
los eventos locales no son independientes porque $M_R(c)$ tiene forma cuadrática rígida; esto introduce un producto convergente
de correcciones de orden $p^{-2}$.
\end{itemize}

Para fijar notación, definimos \emph{heurísticamente} un factor local de primer orden como el límite (o valor base):
\[
\kappa_p^{(0)}(\Delta):=
\begin{cases}
\sigma_{p,\Delta}^{(0)}, & p\ \text{inerte (incl. $p=2$ si aplica)},\\
1, & p\ \text{split},\\
\kappa_p^{\mathrm{ram}}(\Delta), & p\mid \Delta\ \text{(ramificado)}.
\end{cases}
\]
Para $p\ge 3$ inerte, el Teorema~\ref{thm:local-odd} da $\kappa_p^{(0)}(\Delta)=(p-1)/(p+1)$.
Para $p=2$ inerte, el Teorema~\ref{thm:sigma2-inerte} da $\kappa_2^{(0)}(\Delta)=2/3$.
Para la teoría completa, $\kappa_p$ sería la densidad del evento ``$M_R(c)$ es localmente norma'' normalizada por el factor de criba standard.
Los ramificados son finitos y su contribución se absorbe en un factor finito.

\begin{lemma}[Renormalización correcta y convergencia de la parte singular]
\label{lem:kappa-conv}
Sea $\chi=\chi_\Delta$ el carácter de Kronecker. Para todo primo impar $p\nmid \Delta$ definimos el factor de primer orden
\[
\kappa_p^{(0)}(\Delta):=
\begin{cases}
1, & \chi(p)=+1\ (\text{split}),\\[4pt]
\sigma_{p,\Delta}^{(0)}=\dfrac{p-1}{p+1}, & \chi(p)=-1\ (\text{inerte}).
\end{cases}
\]
Entonces vale la identidad exacta
\[
\kappa_p^{(0)}(\Delta)=\frac{1-\frac{1}{p}}{1-\frac{\chi(p)}{p}}
\qquad (p\ge 3,\ p\nmid\Delta).
\]
En particular, el producto impar truncado
\[
P_\Delta^{\mathrm{odd}}(x):=\prod_{3\le p\le x}\frac{1-\frac{1}{p}}{1-\frac{\chi(p)}{p}}
\]
incorpora \emph{exactamente} el primer orden de todos los primos impares no ramificados; toda discrepancia restante en la serie singular
proviene únicamente del conjunto finito $p\mid 2\Delta$ y de correcciones finas de orden $\ge p^{-2}$.
\end{lemma}

\begin{proof}
Para $p\nmid\Delta$ tenemos $\chi(p)=\pm 1$ y por cálculo directo:
si $\chi(p)=+1$ entonces $\frac{1-\frac{1}{p}}{1-\frac{1}{p}}=1$; si $\chi(p)=-1$ entonces
\[
\frac{1-\frac{1}{p}}{1+\frac{1}{p}}=\frac{p-1}{p+1}=\sigma_{p,\Delta}^{(0)}.
\]
Esto prueba la identidad. Como el conjunto de primos $p\mid 2\Delta$ es finito, sus factores contribuyen un multiplicador finito.
Las correcciones verdaderamente ``locales finas'' empiezan en orden $p^{-2}$ (véase \eqref{eq:odd-k-only} y el Teorema~\ref{thm:renorm}),
por lo que el producto residual converge absolutamente.
\end{proof}

\subsection{Una identidad de Euler que selecciona exactamente a los inertes}
Sea $\chi=\chi_\Delta$ el carácter cuadrático de Kronecker. Para $p\nmid\Delta$ tenemos $\chi(p)=\pm 1$.
Obsérvese que para $p\nmid\Delta$ se cumple la identidad exacta
\begin{equation}
\label{eq:local-ratio}
\frac{1-\frac{1}{p}}{1-\frac{\chi(p)}{p}}
=
\begin{cases}
1, & \chi(p)=+1\ \text{(split)},\\[4pt]
\frac{p-1}{p+1}, & \chi(p)=-1\ \text{(inerte)}.
\end{cases}
\end{equation}
Es decir: para primos impares (donde la identidad es exacta con la densidad local), el cociente $(1-1/p)/(1-\chi(p)/p)$ \emph{no ve} a los split y multiplica exactamente el factor $(p-1)/(p+1)$ sobre los inertes.
\begin{remark}[Advertencia sobre el lugar $2$]
Si $p=2$ es inerte, la fórmula \eqref{eq:local-ratio} daría $\frac{1/2}{1+1/2}=1/3$, mientras que la densidad local correcta es $\sigma_{2,\Delta}^{(0)}=2/3$ (Teorema~\ref{thm:sigma2-inerte}).
Por esta razón, la identidad Euleriana debe restringirse a la parte impar, dejando el lugar $2$ como un factor separado (tratado en el capítulo $2$-ádico).
\end{remark}

\subsection{Producto truncado y corrección ramificada (arreglo canónico)}
Para evitar ambigüedades con $p\mid\Delta$ y separar la anomalía en $p=2$, definimos el producto \textbf{impar}:
\[
P_\Delta^{\mathrm{odd}}(x):=\prod_{\substack{3\le p\le x}}\frac{1-\frac{1}{p}}{1-\frac{\chi_\Delta(p)}{p}}.
\]
Cuando $p\mid\Delta$ ($p\ge 3$) se tiene $\chi_\Delta(p)=0$ y el factor correspondiente es simplemente $(1-1/p)$.
Esta definición captura la ``física'' de todos los primos impares. El factor $2$-ádico se inserta explícitamente a posteriori.

\subsection{Aparición inevitable de $e^{-\gamma}$ y del residuo $L(1,\chi_\Delta)$}
Usando productos de Euler,
\[
\prod_{p\le x} \Big(1-\frac{\chi_\Delta(p)}{p}\Big)^{-1}
\longrightarrow L(1,\chi_\Delta)
\qquad (x\to\infty),
\]
mientras que por Mertens
\[
\prod_{p\le x}\Big(1-\frac{1}{p}\Big)\sim \frac{e^{-\gamma}}{\log x}.
\]
Combinando, se obtiene el límite canónico (sin factores finitos extra):
\[
P_\Delta(x):=\prod_{p\le x}\frac{1-\frac{1}{p}}{1-\frac{\chi_\Delta(p)}{p}},
\qquad
P_\Delta^{\mathrm{odd}}(x):=\prod_{3\le p\le x}\frac{1-\frac{1}{p}}{1-\frac{\chi_\Delta(p)}{p}}.
\]
Entonces
\[
P_\Delta(x)=P_\Delta^{\mathrm{odd}}(x)\cdot \frac{1-\frac12}{1-\frac{\chi_\Delta(2)}{2}}
= P_\Delta^{\mathrm{odd}}(x)\cdot \frac{\frac12}{1-\frac{\chi_\Delta(2)}{2}}.
\]

\begin{equation}
\label{eq:lim-mertens-chi}
\boxed{
\lim_{x\to\infty} \log x \cdot P_\Delta(x)= e^{-\gamma}L(1,\chi_\Delta).
}
\end{equation}
Este es exactamente el mecanismo por el cual, en dimensión $1$, el renormalizador $e^{-\gamma}$ es forzado,
y el campo entra únicamente vía $L(1,\chi_\Delta)$ (equivalentemente, vía $\mathrm{Res}_{s=1}\zeta_K(s)$).

\subsection{Expansión logarítmica: por qué emergen $\sum \chi(p)/p^2$ y valores $L(2m,\chi)$}
La identidad \eqref{eq:local-ratio} también permite una expansión uniforme.
Para $p\nmid\Delta$,
\[
\log\frac{1-\frac{1}{p}}{1-\frac{\chi(p)}{p}}
=
\log(1-\tfrac{1}{p})-\log(1-\tfrac{\chi(p)}{p})
=
-\sum_{k\ge 1}\frac{1-\chi(p)^k}{k\,p^k}.
\]
Como $\chi(p)^2=1$ para $p\nmid\Delta$, se tiene $1-\chi(p)^k=0$ si $k$ es par y $1-\chi(p)$ si $k$ es impar.
Luego
\begin{equation}
\label{eq:odd-k-only}
\log\frac{1-\frac{1}{p}}{1-\frac{\chi(p)}{p}}
=
-(1-\chi(p))\sum_{j\ge 0}\frac{1}{(2j+1)p^{2j+1}}
=
-\frac{1-\chi(p)}{p}+O\Big(\frac{1}{p^3}\Big).
\end{equation}
Esto explica el coeficiente ``armónico'' unificado:
\[
\alpha(\chi(p))=-(1-\chi(p)),
\quad \text{es decir: }
\alpha=0 \text{ en split},\ \alpha=-2 \text{ en inerte},\ \alpha=-1 \text{ en ramificado}.
\]

Al sustraer el término principal $\frac{1-\chi(p)}{p}$, el resto es $O(p^{-3})$ y por tanto absolutamente sumable.
Esto permite definir una constante renormalizada \emph{absolutamente convergente}:
\[
\mathcal{M}_\Delta
:=
\prod_{p}
\frac{1-\frac{1}{p}}{1-\frac{\chi(p)}{p}}
\exp\Big(\frac{1-\chi(p)}{p}\Big),
\qquad
\log\mathcal{M}_\Delta=\sum_{p} O(p^{-3}).
\]
La cota anterior se entiende salvo un conjunto finito de primos, donde la corrección puede escribirse como $O(p^{-2})$.
En particular, los términos de orden $p^{-2}$ que aparecen cuando se estudian correcciones más finas
se expresan naturalmente en términos de productos convergentes en $s\ge 2$:
\[
\frac{\zeta(2)}{L(2,\chi_\Delta)}
=
\prod_{p}\frac{1-\frac{\chi_\Delta(p)}{p^2}}{1-\frac{1}{p^2}},
\qquad
\log\frac{\zeta(2)}{L(2,\chi_\Delta)}
=
\sum_{p}\frac{1-\chi_\Delta(p)}{p^2}+O\Big(\sum_p\frac{1}{p^4}\Big).
\]

\begin{remark}[Lectura conceptual]
\eqref{eq:lim-mertens-chi} explica el baseline $e^{-\gamma}L(1,\chi_\Delta)$ (y por ende $2e^{-\gamma}L(1,\chi_\Delta)$ en la constante final).
La identidad con $\zeta(2)/L(2,\chi_\Delta)$ explica por qué las correcciones finas de la serie singular “huelen” a valores de $\zeta$ y $L$
en enteros: las correcciones verdaderamente locales empiezan en orden $p^{-2}$.
\end{remark}

% ================================================================
\section{Evidencia computacional: cómo se estima $\Csinf(\Delta)$ y por qué la cola manda}
\label{sec:computo}
% ================================================================

\subsection{Qué se computa}
Para cada $R$ en una malla (en los experimentos principales: $R=101+20k$, i.e. $R\equiv 1\pmod{20}$, por lo que $R$ es impar y $v_5(R)=0$; los CSV conservan el sufijo histórico ``step10'') se computa:
\[
N_\Delta(R),\qquad H_\Delta(R),\qquad C_{\sigma,\Delta}(R)=\frac{N_\Delta(R)\log R}{H_\Delta(R)R}.
\]
La generación primaria de $N_\Delta(R)$ queda automatizada en un script dedicado,
con CLI reproducible y semilla explícita para los modos estocásticos.
Esta malla principal fija además $v_2(R)=0$, por lo que no explora estratos $2$-ádicos ni estratos con $v_5(R)>0$; esos casos requieren mallas complementarias.
Para ello, añadimos una malla complementaria por estratos exactos
\[
(v_2(R),v_5(R))\in\{(1,0),(2,0),(3,0),(0,1),(1,1),(2,1),(0,2)\},
\]
parametrizada como $R=2^{\alpha}5^{\beta}q$ con $(q,10)=1$.
En el caso de Heegner ($h(\Delta)=1$) es particularmente eficiente usar el principio local--global:
un racional $x\in\Q^\times$ es norma global en $K^\times$ si y solo si es norma local en todo lugar (Proposición~\ref{prop:hasse-norm}).
En los CSV principales fijamos una convención operativa \emph{inerte-only}: el observable se normaliza únicamente con los factores inertes
(incluyendo $p=2$ solo si es inerte y divide a $R$). Para nuestro conteo, esto se traduce en:

\begin{itemize}[leftmargin=2em]
\item Para todo primo impar $p\nmid \Delta$, el criterio local se reduce a la \emph{paridad}:
si $p$ es inerte entonces es necesario y suficiente que $v_p(n)$ sea par; si $p$ es split no hay obstrucción.
\item Los primos ramificados se dejan fuera de $H_\Delta(R)$ en esta convención base y se estudian por separado mediante
el refinamiento $H_\Delta^\star$ (ver \Cref{subsec:ramified-star-tail}).
\end{itemize}

Usamos factorización directa con precomputación de primos (criba de Eratóstenes modificada) para verificar esta paridad eficientemente.

\subsection{Por qué el promedio no es el límite}
En problemas de criba es típico que el acercamiento al límite tenga correcciones del tipo $1/\log R$.
Así, un promedio sobre todo el rango mezcla regímenes finitos con el régimen asintótico y sesga hacia arriba/abajo.
Por eso se usa un ajuste sobre la cola.

\subsection{Ajuste de cola (ansatz de relajación)}
Usamos el modelo empírico estable:
\[
C_{\sigma,\Delta}(R)\approx \Csinf(\Delta)+\frac{a}{\log R}\qquad (R\ \text{grande}),
\]
y estimamos $\Csinf(\Delta)$ por regresión lineal de $C_{\sigma,\Delta}(R)$ contra $1/\log R$
restringiendo a $R\ge 10^6$ (y opcionalmente bindeando en cuantiles de $\log R$ para estabilizar varianza).
Como diagnóstico de sesgo de truncación/modelo, también ajustamos
\[
C_{\sigma,\Delta}(R)\approx \Csinf(\Delta)+\frac{a}{\log R}+\frac{b}{\log^2 R}
\]
en la misma cola y en la rejilla $R_{\mathrm{tail}}\in\{10^6,2\cdot 10^6,3\cdot 10^6\}$.
Así, reportamos dos capas de incertidumbre para $C_\infty$:
\[
\mathrm{EE}_{\mathrm{WLS}} \quad\text{(in-model)},
\qquad
\mathrm{EE}_{\mathrm{sist}}:=\max_{R_{\mathrm{tail}}}\big|C_\infty(R_{\mathrm{tail}})-C_\infty(10^6)\big|.
\]

\begin{table}[htbp]
\centering
\caption{Heegner: baseline canónico $C_{\mathrm{base}}(\Delta)$ y ajuste de cola $C_\infty(\Delta)$ usando $C_{\sigma,\Delta}(R)\approx C_\infty+a/\log R$ en $R\ge 10^6$ (200 bins por cuantiles). Se reporta $\mathrm{EE}_{\mathrm{WLS}}$ (incertidumbre in-model) del WLS en bins (pesos por tamaño de bin).}
\label{tab:heegner-tailfit}
\sisetup{group-separator={},group-minimum-digits=4}
\begin{tabular}{
    r
    S[table-format=1.6]
    S[table-format=1.6]
    S[table-format=1.6]
    S[table-format=-1.3]
    S[table-format=1.3]
    S[table-format=1.6]
}
\toprule
{$\Delta$} & {$C_{\mathrm{base}}$} & {$C_\infty$} & {$\mathrm{EE}_{\mathrm{WLS}}(C_\infty)$} & {$a$} & {$\mathrm{EE}_{\mathrm{WLS}}(a)$} & {$\mathfrak{S}_\Delta$} \\
\midrule
$-163$ & 0.276315 & 0.156052 & 0.000063 & -0.010 & 0.001 & 0.564763 \\
$-67$  & 0.430984 & 0.241297 & 0.000107 & 0.272  & 0.002 & 0.559875 \\
$-43$  & 0.537977 & 0.303469 & 0.000140 & 0.400  & 0.002 & 0.564093 \\
$-19$  & 0.809322 & 0.473185 & 0.000208 & 0.590  & 0.003 & 0.584669 \\
$-11$  & 1.063658 & 0.650708 & 0.000281 & 0.666  & 0.004 & 0.611765 \\
$-8$   & 1.247249 & 0.692068 & 0.000070 & 0.819  & 0.001 & 0.554875 \\
$-7$   & 1.333366 & 0.431485 & 0.000186 & 0.513  & 0.003 & 0.323606 \\
$-4$   & 0.881938 & 0.735033 & 0.000064 & 0.775  & 0.001 & 0.833428 \\
$-3$   & 0.678917 & 0.566005 & 0.000065 & 0.563  & 0.001 & 0.833689 \\
\bottomrule
\end{tabular}
\end{table}

\begin{table}[htbp]
\centering
\caption{Descomposición de incertidumbre de cola para $C_\infty(\Delta)$.
Modelo de referencia: $C_\infty+a/\log R$ en $R_{\mathrm{tail}}=10^6$.
Se reporta $\mathrm{EE}_{\mathrm{WLS}}$, la sensibilidad de truncación $\mathrm{EE}_{\mathrm{sist}}$ en
$R_{\mathrm{tail}}\in\{10^6,2\cdot 10^6,3\cdot 10^6\}$, y el desplazamiento de modelo
$|\Delta_{\mathrm{model}}|:=|C_\infty^{(2)}-C_\infty^{(1)}|$ entre
$C_\infty+a/\log R+b/\log^2 R$ y $C_\infty+a/\log R$ en $R_{\mathrm{tail}}=10^6$.
Además, $\mathrm{EE}_{\mathrm{full}}:=\sqrt{\mathrm{EE}_{\mathrm{WLS}}^2+\mathrm{EE}_{\mathrm{sist}}^2+|\Delta_{\mathrm{model}}|^2}$.}
\label{tab:tail-stability}
\sisetup{group-separator={},group-minimum-digits=4}
\begin{tabular}{r S[table-format=1.6] S[table-format=1.6] S[table-format=1.6] S[table-format=1.6] S[table-format=1.6]}
\toprule
{$\Delta$} & {$C_\infty^{(1)}$} & {$\mathrm{EE}_{\mathrm{WLS}}$} & {$\mathrm{EE}_{\mathrm{sist}}$} & {$|\Delta_{\mathrm{model}}|$} & {$\mathrm{EE}_{\mathrm{full}}$} \\
\midrule
$-163$ & 0.156052 & 0.000063 & 0.000530 & 0.004047 & 0.004082 \\
$-67$  & 0.241297 & 0.000107 & 0.000401 & 0.006613 & 0.006626 \\
$-43$  & 0.303469 & 0.000140 & 0.001303 & 0.022085 & 0.022124 \\
$-19$  & 0.473185 & 0.000208 & 0.001055 & 0.017593 & 0.017626 \\
$-11$  & 0.650708 & 0.000281 & 0.001388 & 0.022101 & 0.022146 \\
$-8$   & 0.692068 & 0.000070 & 0.001398 & 0.019602 & 0.019652 \\
$-7$   & 0.431485 & 0.000186 & 0.000786 & 0.011712 & 0.011740 \\
$-4$   & 0.735032 & 0.000064 & 0.001259 & 0.019624 & 0.019665 \\
$-3$   & 0.566005 & 0.000065 & 0.001074 & 0.016527 & 0.016562 \\
\bottomrule
\end{tabular}
\end{table}
En toda la familia se observa $\mathrm{EE}_{\mathrm{sist}}>\mathrm{EE}_{\mathrm{WLS}}$ (entre $3.8\times$ y $19.9\times$),
pero además domina el término de modelo:
$\mathrm{EE}_{\mathrm{full}}/\sqrt{\mathrm{EE}_{\mathrm{WLS}}^2+\mathrm{EE}_{\mathrm{sist}}^2}\in[7.65,16.88]$,
lo que confirma que la metrología de $C_\infty$ debe separar incertidumbre estadística in-model y sensibilidad sistemática de cola/modelo.

\subsection{Refinamiento ramificado en cola: $H_\Delta^\star$}
\label{subsec:ramified-star-tail}
Sobre los mismos CSV aplicamos el post-procesado
\[
H_\Delta^\star(R)=H_\Delta(R)\prod_{\substack{p\mid \Delta\\ p\ \text{impar}}}p^{-v_p(R)},
\qquad
C_{\sigma,\Delta}^\star(R):=\frac{N_\Delta(R)\log R}{H_\Delta^\star(R)\,R}.
\]
Equivalentemente:
\[
C_{\sigma,\Delta}^\star(R)=C_{\sigma,\Delta}(R)\prod_{\substack{p\mid \Delta\\ p\ \text{impar}}}p^{v_p(R)}.
\]
La reducción de varianza en la cola es sustancial:
\begin{table}[htbp]
\centering
\caption{Efecto del refinamiento ramificado en la cola $R\in[10^6,10^7]$.}
\label{tab:tail-star-sd}
\sisetup{group-separator={},group-minimum-digits=4}
\begin{tabular}{r S[table-format=1.3] S[table-format=1.3] S[table-format=1.3]}
\toprule
{$\Delta$} & {$\operatorname{SD}(C_{\sigma,\Delta})$} & {$\operatorname{SD}(C_{\sigma,\Delta}^\star)$} & {ratio} \\
\midrule
$-3$  & 0.279 & 0.072 & 0.259 \\
$-7$  & 0.159 & 0.045 & 0.283 \\
$-11$ & 0.196 & 0.058 & 0.296 \\
$-43$ & 0.049 & 0.021 & 0.423 \\
\bottomrule
\end{tabular}
\end{table}
Este comportamiento confirma que separar explícitamente el bloque ramificado mejora la estabilidad estadística del ajuste de cola.

\begin{remark}
Los casos $\Delta=-4,-8$ (pares) y $\Delta=-163$ exhiben estabilidad extrema en la cola, mientras que $\Delta=-3$ presenta varianza residual alta.
Esto es consistente con el hecho de que en $\Delta=-3$ interviene además el primo ramificado $3$ (Apéndice \Cref{subsec:sigma3-minus3})
y con la dominancia de primos pequeños en la serie singular.
\end{remark}

\begin{remark}[Lectura inmediata]
Para $\Delta=-43,-67,-163$ (donde $2$ es inerte) la relación $\Ssing\approx 0.56$ se puede interpretar como producto de un
factor $2$-ádico explícito (Teorema~\ref{thm:sigma2-inerte}) y un factor impar $\Sodd$ (ver \Cref{sec:renorm} y \Cref{sec:zetaL}).
\end{remark}

\subsection{Estabilización visual de $C_{\sigma,\Delta}(R)$}
La \Cref{fig:stabilization-all} muestra la convergencia de $C_{\sigma,\Delta}(R)$ hacia el límite $\Csinf(\Delta)$
para los nueve discriminantes de Heegner. La línea roja corresponde al ajuste de cola
$C_{\sigma,\Delta}(R)\approx C_\infty(\Delta)+a/\log R$ en la región $R\ge 10^6$.

\begin{figure}[htbp]
\centering
\begin{subfigure}[b]{0.32\linewidth}
\incfig[width=\linewidth]{../../Figuras/Articulo/D-3_stabilizacion.pdf}
\caption{$\Delta=-3$}
\end{subfigure}
\hfill
\begin{subfigure}[b]{0.32\linewidth}
\incfig[width=\linewidth]{../../Figuras/Articulo/D-4_stabilizacion.pdf}
\caption{$\Delta=-4$}
\end{subfigure}
\hfill
\begin{subfigure}[b]{0.32\linewidth}
\incfig[width=\linewidth]{../../Figuras/Articulo/D-7_stabilizacion.pdf}
\caption{$\Delta=-7$}
\end{subfigure}
\\[1em]
\begin{subfigure}[b]{0.32\linewidth}
\incfig[width=\linewidth]{../../Figuras/Articulo/D-8_stabilizacion.pdf}
\caption{$\Delta=-8$}
\end{subfigure}
\hfill
\begin{subfigure}[b]{0.32\linewidth}
\incfig[width=\linewidth]{../../Figuras/Articulo/D-11_stabilizacion.pdf}
\caption{$\Delta=-11$}
\end{subfigure}
\hfill
\begin{subfigure}[b]{0.32\linewidth}
\incfig[width=\linewidth]{../../Figuras/Articulo/D-19_stabilizacion.pdf}
\caption{$\Delta=-19$}
\end{subfigure}
\\[1em]
\begin{subfigure}[b]{0.32\linewidth}
\incfig[width=\linewidth]{../../Figuras/Articulo/D-43_stabilizacion.pdf}
\caption{$\Delta=-43$}
\end{subfigure}
\hfill
\begin{subfigure}[b]{0.32\linewidth}
\incfig[width=\linewidth]{../../Figuras/Articulo/D-67_stabilizacion.pdf}
\caption{$\Delta=-67$}
\end{subfigure}
\hfill
\begin{subfigure}[b]{0.32\linewidth}
\incfig[width=\linewidth]{../../Figuras/Articulo/D-163_stabilizacion.pdf}
\caption{$\Delta=-163$}
\end{subfigure}
\caption{Estabilización de $C_{\sigma,\Delta}(R)$ vs $R$ (escala log) para los nueve discriminantes de Heegner. Línea roja: ajuste de cola $C_{\sigma,\Delta}(R)\approx C_\infty(\Delta)+a/\log R$ para $R\ge 10^6$.}
\label{fig:stabilization-all}
\end{figure}
\FloatBarrier

\subsection{Comparativa Heegner: el factor singular}
La \Cref{fig:heegner-comparative} muestra la relación entre la constante empírica $\Csinf(\Delta)$ y el baseline teórico $\Csbase(\Delta)$.
El cociente $\Ssing=\Csinf/\Csbase$ captura el factor singular de la familia.

\begin{figure}[htbp]
\centering
\begin{subfigure}[b]{0.48\linewidth}
\incfig[width=\linewidth]{../../Figuras/Articulo/Heegner_Ssing.pdf}
\caption{Factor singular $\Ssing=\Csinf/\Csbase$}
\end{subfigure}
\hfill
\begin{subfigure}[b]{0.48\linewidth}
\incfig[width=\linewidth]{../../Figuras/Articulo/Heegner_Cinf_vs_Cbase.pdf}
\caption{Scatter $\Csinf$ vs $\Csbase$}
\end{subfigure}
\caption{Comparativa Heegner: (a) barra de $\Ssing$ mostrando el outlier $\Delta=-7$; (b) relación entre constante empírica y baseline teórico.}
\label{fig:heegner-comparative}
\end{figure}
\FloatBarrier

\subsection{Colapso lineal: $C_{\rm raw}(R)$ vs $H_\Delta(R)$}
Si $C_{\sigma,\Delta}(R)$ se estabiliza hacia $\Csinf(\Delta)$, entonces por la ley global
(Conjetura~\ref{conj:global}, o Teorema~\ref{thm:global-conditional} bajo hipótesis analítica):
\[
C_{\rm raw}(R)=\frac{N_\Delta(R)\log R}{R}\approx \Csinf(\Delta)\,H_\Delta(R).
\]
La \Cref{fig:D43-collapse} confirma este colapso lineal: al graficar $C_{\rm raw}$ contra $H_\Delta$, la nube se alinea,
demostrando que $H_\Delta$ no es un ``fit hack'' sino estructura.

\begin{figure}[htbp]
\centering
\incfig[width=0.85\linewidth]{../../Figuras/Articulo/D-43_Craw_vs_H.pdf}
\caption{Colapso lineal: $C_{\rm raw}(R)$ vs $H_\Delta(R)$ para $\Delta=-43$. El alineamiento demuestra que $H_\Delta$ captura la oscilación determinista.}
\label{fig:D43-collapse}
\end{figure}
\FloatBarrier

% ================================================================
\section{Ejemplo de control: el outlier $\Delta=-7$ explicado por inertes pequeños}
\label{sec:ejemplo7}
% ================================================================

\begin{example}[Producto de pequeños inertes y predicción de primer orden]
\label{ex:minus7}
Para $\Delta=-7$,
\[
\Csbase(-7)=\frac{4\pi e^{-\euler}}{2\sqrt 7}\approx 1.333366.
\]
En $\Q(\sqrt{-7})$, los primos $3$ y $5$ son inertes. Si restringimos a $R$ coprimo con $15$ (de modo que $v_3(R)=v_5(R)=0$),
entonces el Teorema~\ref{thm:local-odd} da
\[
\sigma_3^{(0)}=\frac{3-1}{3+1}=\frac12,
\qquad
\sigma_5^{(0)}=\frac{5-1}{5+1}=\frac23.
\]
El producto de primer orden es
\[
\sigma_3^{(0)}\sigma_5^{(0)}=\frac{1}{3},
\]
y por tanto la predicción inmediata es
\[
\Csinf(-7)\approx \Csbase(-7)\cdot \frac{1}{3}\approx 0.444.
\]
Comparado con $\Csinf(-7)\approx 0.431$ (tabla), la diferencia restante es compatible con correcciones de primos mayores y efectos finitos:
lo crucial es que el ``desplome'' se predice por estructura local, no por accidente.
\end{example}

La \Cref{fig:partial-products} muestra el producto parcial de primer orden sobre primos inertes:
\[
\prod_{\substack{p\le x \\ \chi_\Delta(p)=-1}}\frac{p-1}{p+1}.
\]
Para $\Delta=-7$, este producto cae rápidamente debido a los pocos inertes pequeños ($p=3,5$), explicando el outlier.

\begin{figure}[htbp]
\centering
\incfig[width=0.75\linewidth]{../../Figuras/Articulo/Heegner_partial_products.pdf}
\caption{Producto parcial de primer orden sobre primos inertes para los nueve discriminantes de Heegner. La caída de $\Delta=-7$ se explica por pocos inertes pequeños ($p=3,5$).}
\label{fig:partial-products}
\end{figure}

Además, la \Cref{fig:D7-vp-effects} muestra la dependencia respecto de $v_3(R)$.
En la malla principal ($R\equiv 1\pmod{20}$), se tiene $v_5(R)=0$ para todo $R$, de modo que el panel de $p=5$
funciona como control (sin estratificación adicional). Esta dependencia en $v_3(R)$ es exactamente la que captura
el factor inercial $H_\Delta(R)$.

\begin{figure}[htbp]
\centering
\begin{subfigure}[b]{0.48\linewidth}
\incfig[width=\linewidth]{../../Figuras/Articulo/D-7_vp_effect_p3.pdf}
\caption{Efecto de $v_3(R)$}
\end{subfigure}
\hfill
\begin{subfigure}[b]{0.48\linewidth}
\incfig[width=\linewidth]{../../Figuras/Articulo/D-7_vp_effect_p5.pdf}
\caption{Control en $p=5$ (malla: $v_5(R)=0$)}
\end{subfigure}
\caption{Dependencia del observable $C_{\sigma,-7}(R)$ respecto de la valuación de $R$ para inertes pequeños. Izquierda: estratificación efectiva por $v_3(R)$. Derecha: control para $p=5$ en la malla $R\equiv1\pmod{20}$ (no hay estratos con $v_5(R)>0$).}
\label{fig:D7-vp-effects}
\end{figure}

\subsection{Predicción de primer orden por primos inertes pequeños}
La observación clave del Ejemplo~\ref{ex:minus7} se puede cuantificar sistemáticamente: para cada discriminante de Heegner,
usamos solo los primos inertes $p\le 50$ para construir una predicción de primer orden:
\[
C_\infty^{\text{pred}}(\Delta):=\Csbase(\Delta)\cdot\prod_{\substack{3\le p\le 50 \\ \chi_\Delta(p)=-1}}\frac{p-1}{p+1}.
\]
La \Cref{fig:pred-small-inerts} compara esta predicción con los valores observados $\Csinf(\Delta)$ obtenidos del ajuste de cola.
El patrón es monótono (predicciones mayores tienden a corresponder a observaciones mayores), pero con subestimación sistemática de escala:
el primer orden con inertes pequeños captura parte de la variación relativa entre discriminantes, mientras que la normalización completa
requiere factores adicionales (incluyendo el capítulo $2$-ádico y correcciones finas de segundo orden $O(p^{-2})$).

\begin{figure}[htbp]
\centering
\incfig[width=0.75\linewidth]{../../Figuras/Articulo/Extra/Heegner_pred_small_inerts.pdf}
\caption{Comparación entre la predicción de primer orden $C_\infty^{\text{pred}}$ (usando solo primos inertes $p\le 50$) y el valor observado $\Csinf(\Delta)$ para los nueve discriminantes de Heegner. La nube conserva orden relativo, pero evidencia subestimación de escala cuando se usa solo primer orden.}
\label{fig:pred-small-inerts}
\end{figure}
\FloatBarrier

% ================================================================
\section{Regularización tipo Mertens: por qué la serie singular debe converger}
\label{sec:renorm}
% ================================================================

\begin{theorem}[Renormalización universal con coeficientes variables]
\label{thm:renorm}
Sea $(a_p)_p$ una familia indexada por primos tal que
\[
a_p=\frac{\alpha_p}{p}+O\Big(\frac{1}{p^2}\Big)\qquad (p\to\infty),
\]
con $(\alpha_p)_p$ una sucesión complejamente acotada. 
Entonces, para $p$ suficientemente grande (asumiendo $|a_p|\le 1/2$ y $1+a_p\neq 0$), el producto renormalizado
\[
\prod_p(1+a_p)\exp(-\alpha_p/p)
\]
converge absolutamente (salvo anulación de un número finito de factores) porque
\[
\log\big((1+a_p)e^{-\alpha_p/p}\big)=O(p^{-2}).
\]
\end{theorem}
\begin{proof}
Para $|x|$ pequeño, $\log(1+x)=x+O(x^2)$. Como $a_p=O(1/p)$, para $p$ grande:
\[
\log\big((1+a_p)e^{-\alpha_p/p}\big)=\log(1+a_p)-\alpha_p/p
=\Big(\alpha_p/p+O(p^{-2})\Big)-\alpha_p/p+O(p^{-2})=O(p^{-2}).
\]
Como $\sum_p p^{-2}<\infty$, la suma de logaritmos converge absolutamente.
\end{proof}

\begin{corollary}[Convergencia absoluta de la serie singular normalizada]
\label{cor:ssing-absconv}
Con la normalización de \Cref{sec:zetaL}, supongamos que para $p$ impar grande
\[
\Prob_{c\in \Z/p^k\Z}\!\big(M_R(c)\in N(\mathcal O_{K,p})\big)
=(1-1/p)(1-\chi_\Delta(p)/p)^{-1}\bigl(1+b_p(\Delta)\bigr),
\]
con $b_p(\Delta)=O(p^{-2})$ uniforme en $p$ (para $k$ suficientemente grande).
Entonces
\[
\kappa_p(\Delta)=1+b_p(\Delta),
\qquad
\log \kappa_p(\Delta)=O(p^{-2}),
\]
y en consecuencia el producto
\[
\Ssing=\kappa_2(\Delta)\prod_{p\ge 3}\kappa_p(\Delta)
\]
converge absolutamente.
\end{corollary}
\begin{proof}
Basta aplicar el Teorema~\ref{thm:renorm} con $\alpha_p=0$ y $a_p=b_p(\Delta)$.
Entonces $\log(1+b_p(\Delta))=O(p^{-2})$, y como $\sum_p p^{-2}<\infty$ se obtiene convergencia absoluta de
la suma de logaritmos, luego del producto.
\end{proof}

\begin{remark}[Aplicación a $\Ssing$]
La parte de primer orden $(1-1/p)(1-\chi_\Delta(p)/p)^{-1}$ ya está absorbida en el baseline
$2e^{-\euler}L(1,\chi_\Delta)$; la serie singular $\Ssing$ recoge únicamente la corrección fina residual.
\end{remark}

% ================================================================
\section{Discusión final: manuscrito blindado y siguiente paso hacia un teorema global}
\label{sec:discusion}
% ================================================================

\subsection{Lo que ya está ``cerrado''}
El corazón del paper es local y está completamente sellado:
\begin{itemize}[leftmargin=2em]
\item Para $p\ge 3$ (inerte), la densidad local es exactamente $\sigma_p^{(m)}=1-\frac{2}{p^m(p+1)}$ (Teorema~\ref{thm:local-odd}).
\item Para $p=2$ inerte, la densidad es $\sigma_2^{(m)}=1-\frac{1}{3\cdot 2^m}$ (Teorema~\ref{thm:sigma2-inerte}),
y en particular para $R$ impar aparece el racional fijo $2/3$.
\item El baseline canónico del campo es $\Csbase(\Delta)=2e^{-\euler}L(1,\chi_\Delta)$
(Teorema~\ref{thm:arch}), y en Heegner se vuelve cerradamente $\frac{4\pi e^{-\euler}}{w(\Delta)\sqrt{|\Delta|}}$.
\item Los outliers se explican por productos de pequeños inertes (Ejemplo~\ref{ex:minus7}).
\end{itemize}

\subsection{De la Proposición~12.3 al Teorema~12.1: del tamiz al contorno de Henriot}
\label{subsec:12-bridge}

Recordemos que queremos estimar
\[
N_\Delta(R)\;=\;\#\{1\le c\le R:\; M_R(c)=R^2-c^2 \text{ es una norma en }K=\mathbb{Q}(\sqrt{\Delta})\}.
\]
Sea $f_\Delta:\mathbb{N}\to\{0,1\}$ la función multiplicativa indicadora de la condición local de norma:
para cada primo $p\nmid \Delta$,
\[
f_\Delta(p^k)=
\begin{cases}
1,& \chi_\Delta(p)\in\{0,+1\}\ \text{(ramificado o split) y }k\ge 0,\\
1,& \chi_\Delta(p)=-1\ \text{(inerte) y }k\ \text{par},\\
0,& \chi_\Delta(p)=-1\ \text{y }k\ \text{impar},
\end{cases}
\]
y extendemos multiplicativamente (con las modificaciones locales finitas necesarias cuando $p\mid \Delta$).
Entonces
\[
N_\Delta(R)=\sum_{1\le c\le R} f_\Delta\!\bigl(M_R(c)\bigr).
\]

\medskip
\noindent\textbf{(i) El rol exacto de la Proposición~12.3.}
Para aplicar tanto un tamiz de Selberg como el marco de Henriot--Nair--Tenenbaum, la entrada
indispensable es un control uniforme de la distribución de la sucesión $M_R(c)$ en clases módulo $q$,
al menos para $q$ libre de cuadrados y en el caso de divisibilidad ($a=0$).
Definimos, para $q$ squarefree,
\[
\rho_R(q)\;=\;\#\{x\bmod q:\; M_R(x)\equiv 0\!\!\pmod q\}.
\]
La Proposición~12.3 da precisamente:
\begin{equation}\label{eq:12-bridge-maincount}
\#\{1\le c\le R:\ q\mid M_R(c)\}
=\frac{R}{q}\rho_R(q)+O\!\bigl(\rho_R(q)\bigr),
\qquad \rho_R(q)\ll 2^{\omega(q)}.
\end{equation}
En particular, para todo $\theta<1$ y todo $A>0$,
\begin{equation}\label{eq:12-bridge-level}
\sum_{\substack{q\le R^\theta\\ q\ \mathrm{squarefree}}}
\max_{(a,q)=1}\bigl|E(R;q,a)\bigr|
\ \ll_A\ \frac{R}{(\log R)^A},
\end{equation}
ya que el error puntual está dominado por $\rho_R(q)$ y
$\sum_{q\le R^\theta}2^{\omega(q)}\ll R^\theta(\log R)^{O(1)}$.
Esto verifica (en la forma estrictamente necesaria para nuestra criba) la hipótesis de ``nivel de distribución''
que aparece como $D_\theta$ en el Teorema~12.1 y, al mismo tiempo, es el control requerido para acotar
los términos de resto en el teorema de Henriot aplicado a productos de formas lineales.

\medskip
\noindent\textbf{(ii) Reducción al caso ``genérico'' y aparición de $H_\Delta(R)$.}
Escribimos
\[
M_R(c)=(R-c)(R+c),
\qquad d(c):=\gcd(R-c,R+c)=\gcd(R-c,2R).
\]
Por tanto, toda correlación local entre los dos factores lineales está forzada por primos $p\mid 2R$.
Los cálculos $p$-ádicos de las Secciones~5--6 muestran que la contribución no genérica de dichos primos
se encapsula en el factor local acumulado $H_\Delta(R)$, definido como el producto de las densidades locales
$\sigma_p^{(v_p(R))}$ (incluyendo el caso $p=2$), es decir,
\begin{equation}\label{eq:12-bridge-H}
H_\Delta(R)=\prod_{p\mid 2R}\sigma_p^{(v_p(R))}(\Delta).
\end{equation}
De este modo, una vez fijado $R$, podemos separar el problema en:
\begin{itemize}
\item primos $p\mid 2R$, tratados exactamente vía \eqref{eq:12-bridge-H};
\item primos $p\nmid 2R$, donde los dos factores $R-c$ y $R+c$ se comportan ``genéricamente''
y el promedio viene gobernado por una teoría de funciones multiplicativas en valores polinomiales.
\end{itemize}

\medskip
\noindent\textbf{(iii) Aplicación del teorema de Henriot (Nair--Tenenbaum generalizado).}
Sea $F_R(n)=(R-n)(R+n)$, producto de dos formas lineales primitivas y no proporcionales.
Bajo las hipótesis estándar de Henriot para funciones multiplicativas no negativas de crecimiento suave
(y usando \eqref{eq:12-bridge-level} para controlar los restos en progresiones módulo $q$ squarefree),
el resultado de \cite{Henriot2012} implica que existe una constante $C_\infty(\Delta)>0$ tal que
\begin{equation}\label{eq:12-bridge-henriot-asymp}
\sum_{1\le c\le R} f_\Delta\!\bigl(F_R(c)\bigr)
=
C_\infty(\Delta)\,\frac{H_\Delta(R)\,R}{\log R}
+O_\Delta\!\left(\frac{H_\Delta(R)\,R}{(\log R)^{1+\eta}}\right)
\end{equation}
para algún $\eta>0$.
La estructura $R/\log R$ corresponde a una \emph{dimensión de criba global} $\kappa=1$:
cada factor lineal aporta $\kappa=\tfrac12$ (densidad de primos inertes) y, en conjunto, $\kappa=1$.

\medskip
\noindent\textbf{(iv) Dónde entra el contorno: forma Selberg--Delange/Perron.}
Una forma conveniente de ver \eqref{eq:12-bridge-henriot-asymp} (y de identificar la constante)
es escribir, para $\Re(s)>1$, la serie de Dirichlet local-global asociada a la sucesión:
\[
\mathcal{F}_R(s)
:=\sum_{n\ge 1}\frac{a_R(n)}{n^s},
\qquad
a_R(n):=\sum_{\substack{1\le c\le R\\ F_R(c)=n}} f_\Delta(n).
\]
Por inversión de Perron,
\begin{equation}\label{eq:12-bridge-perron}
\sum_{1\le c\le R} f_\Delta(F_R(c))
=
\frac{1}{2\pi i}\int_{\sigma-iT}^{\sigma+iT}\mathcal{F}_R(s)\,\frac{R^s}{s}\,ds
+O\!\left(\frac{R^\sigma}{T}\right),
\qquad (\sigma>1).
\end{equation}
El teorema de Henriot (en su demostración) equivale a exhibir una factorización
\[
\mathcal{F}_R(s)
=
\frac{G_\Delta(s;R)}{(s-1)^{\kappa}}
\qquad (\kappa=1),
\]
donde $G_\Delta(s;R)$ es holomorfa y no nula en una vecindad de $s=1$ (uniformemente en $R$),
y donde el aporte no genérico de $p\mid 2R$ queda absorbido en $G_\Delta(1;R)=H_\Delta(R)\cdot G_\Delta(1;\infty)$.
Desplazando el contorno en \eqref{eq:12-bridge-perron} hacia $\Re(s)=1-\varepsilon$ y recogiendo el residuo en $s=1$,
obtenemos
\[
\sum_{1\le c\le R} f_\Delta(F_R(c))
=
\operatorname{Res}_{s=1}\left(\mathcal{F}_R(s)\frac{R^s}{s}\right)
+O_\Delta\!\left(\frac{H_\Delta(R)\,R}{(\log R)^{1+\eta}}\right)
=
G_\Delta(1;R)\,\frac{R}{\log R}
+O_\Delta\!\left(\frac{H_\Delta(R)\,R}{(\log R)^{1+\eta}}\right),
\]
que es \eqref{eq:12-bridge-henriot-asymp} con
\[
C_\infty(\Delta)=G_\Delta(1;\infty).
\]

\medskip
\noindent\textbf{(v) Identificación explícita de $C_\infty(\Delta)$.}
Finalmente, la evaluación local--global desarrollada en las Secciones~7--11 muestra que
\[
C_\infty(\Delta)
=
2e^{-\gamma}\,L(1,\chi_\Delta)\cdot \kappa_2(\Delta)\prod_{p\ge 3}\kappa_p(\Delta),
\]
donde los factores $\kappa_p(\Delta)=1+O(p^{-2})$ son las correcciones finas (producto absolutamente convergente)
y $\kappa_2(\Delta)$ recoge el término $2$-ádico (Sección~6).
Esto completa la transición: la Proposición~12.3 suministra el nivel de distribución requerido,
Henriot da la asintótica con constante Euleriana, y el contorno (Perron + desplazamiento) extrae el residuo en $s=1$,
produciendo el término principal $H_\Delta(R)\,R/\log R$ y dejando un error de potencia logarítmica.

\begin{theorem}[Asintótica global de normas (condicional)]
\label{thm:global-conditional}
Sea $\Delta<0$ un discriminante fundamental. Si la hipótesis $\mathcal D_\theta$ anterior vale para algún $\theta>0$, entonces existe
$\delta>0$ tal que
\[
N_\Delta(R)=\Csinf(\Delta)\,H_\Delta(R)\,\frac{R}{\log R}
+O\!\left(\frac{R}{(\log R)^{1+\delta}}\right),
\]
con
\[
\Csinf(\Delta)=2e^{-\euler}L(1,\chi_\Delta)\,
\kappa_2(\Delta)\prod_{p\ge 3}\kappa_p(\Delta).
\]
\end{theorem}
\begin{proof}[Esquema de demostración]
La hipótesis $\mathcal D_\theta$ entrega la uniformidad en progresiones necesaria para ejecutar una criba de dimensión $1$ con ahorro en el error,
y la reducción al término principal queda descrita en \Cref{subsec:12-bridge}. Los Teoremas~\ref{thm:local-odd} y \ref{thm:sigma2-inerte}
fijan los factores locales, mientras el Teorema~\ref{thm:arch} fija el baseline arquimediano $2e^{-\euler}L(1,\chi_\Delta)$.
La convergencia absoluta del producto residual se obtiene por \Cref{cor:ssing-absconv}; con ello, el término principal queda determinado
y el resto hereda ahorro logarítmico $O(R/(\log R)^{1+\delta})$.
\end{proof}

En síntesis: la estructura de la constante global está completamente determinada por el análisis local del manuscrito;
el único faltante para el resultado incondicional es probar $\mathcal D_\theta$.

% ================================================================
\appendix

\section{Constantes y valores de referencia}
\label{app:const}
\begin{itemize}[leftmargin=2em]
\item $\euler=0.5772156649\ldots$, $e^{-\euler}=0.5614594836\ldots$.
\item En Heegner: $w(-3)=6$ y $w(-4)=4$.
\item Para $\Delta\in\{-7,-8,-11,-19,-43,-67,-163\}$ se tiene $w(\Delta)=2$.
\end{itemize}

% ================================================================
\section{Análisis local explícito en $p=2$ para discriminantes especiales}
\label{app:local2-special}
% ================================================================

\subsection{El lugar $2$ en $K=\mathbb{Q}(\sqrt{-2})$ ($\Delta=-8$): densidad local cerrada}
\label{subsec:sigma2-minus8}

En el completamiento $K_2=\mathbb{Q}_2(\sqrt{-2})$ la norma es
\[
N(a+b\sqrt{-2})=a^2+2b^2.
\]
En particular, para unidades $u\in\mathbb{Z}_2^\times$ se tiene la caracterización clásica:
\begin{equation}\label{eq:norm-units-minus8}
u\in N(K_2^\times)\quad\Longleftrightarrow\quad u\equiv 1,3\pmod 8.
\end{equation}

\begin{theorem}[Densidad local $2$-ádica para $\Delta=-8$]
\label{thm:sigma2-minus8}
Sea $R\in\mathbb{Z}$ y $m=v_2(R)$. Para $c$ distribuido uniformemente respecto a la medida de Haar en $\mathbb{Z}_2$, definimos
\[
\sigma_{2,-8}^{(m)}:=\mathbb{P}\big(M_R(c)=(R-c)(R+c)\in N(K_2^\times)\cup\{0\}\big).
\]
Entonces:
\begin{equation}\label{eq:sigma2-minus8-formula}
\sigma_{2,-8}^{(m)}=
\begin{cases}
\frac12, & m=0\ \ (R\ \text{impar}),\\[6pt]
\frac{3}{2^{m+1}}, & m\ge 1.
\end{cases}
\end{equation}
En particular, en el régimen $R$ impar, el factor local fijo es
\[
\kappa_2(\Delta=-8)=\sigma_{2,-8}^{(0)}=\frac12.
\]
\end{theorem}

\begin{proof}
Usaremos \eqref{eq:norm-units-minus8} y el hecho de que $2$ es norma en $K_2$ (pues $2=N(\sqrt{-2})$), de modo que
la condición de ser norma depende únicamente de la clase módulo $8$ de la \emph{unidad impar} asociada a $M_R(c)$ tras extraer su potencia máxima de $2$.

\smallskip
\noindent\emph{Caso $m=0$ ($R$ impar).}
Partimos según la paridad de $c$.

\smallskip
\noindent\underline{(i) $c$ par.} Escribimos $c=2k$. Entonces $M_R(c)=R^2-4k^2$ es impar y, módulo $8$, como $R^2\equiv 1\pmod 8$,
\[
M_R(c)\equiv 1-4k^2\equiv
\begin{cases}
1\pmod 8,& k\ \text{par},\\
5\pmod 8,& k\ \text{impar}.
\end{cases}
\]
Como $k$ es par/impar con probabilidad $1/2$ bajo Haar, por \eqref{eq:norm-units-minus8} la probabilidad condicional de ser norma es $1/2$.
Por tanto
\[
\mathbb{P}(M_R(c)\ \text{norma}\mid c\ \text{par})=\frac12.
\]

\smallskip
\noindent\underline{(ii) $c$ impar.} Escribimos $c=1+2t$ con $t\in\mathbb{Z}_2$ Haar-uniforme. Entonces
\[
M_R(c)=(R-c)(R+c)=4\cdot \Big(\frac{R-c}{2}\Big)\Big(\frac{R+c}{2}\Big)=:4ab,
\]
donde $a=(R-c)/2$ y $b=(R+c)/2$ satisfacen $a+b=R$ (impar), por lo que exactamente uno de $a,b$ es par y el otro impar.
Sea $E$ el factor par entre $\{a,b\}$ y $O$ el factor impar. Entonces $ab=EO$. Como $O$ es impar,
multiplicar por $O$ permuta las clases de unidades módulo $8$, y por tanto (usando \eqref{eq:norm-units-minus8}) basta analizar la unidad impar
asociada a $E$ tras extraer su potencia de $2$.

Condicionado al evento “$E$ es el factor par”, la variable $E$ es Haar-uniforme en $2\mathbb{Z}_2$,
luego $E/2$ es Haar-uniforme en $\mathbb{Z}_2$ y su unidad impar es uniforme en $\{1,3,5,7\}\pmod 8$.
Por tanto la probabilidad de caer en $\{1,3\}$ es exactamente $1/2$, y obtenemos
\[
\mathbb{P}(M_R(c)\ \text{norma}\mid c\ \text{impar})=\frac12.
\]
Finalmente,
\[
\sigma_{2,-8}^{(0)}=\frac12\cdot\frac12+\frac12\cdot\frac12=\frac12.
\]

\smallskip
\noindent\emph{Caso $m\ge 1$ ($R$ par).}
Escribimos $R=2^m r$ con $r$ impar y descomponemos por capas $t=v_2(c)$.

\smallskip
\noindent\underline{(1) Capa $t=0$ ( $c$ impar ).}
Entonces $R^2\equiv 0\pmod 4$ y $c^2\equiv 1\pmod 8$, así que $M_R(c)=R^2-c^2$ es impar y
\[
M_R(c)\equiv R^2-1\pmod 8.
\]
Si $m=1$, $R^2=4r^2\equiv 4\pmod 8$, luego $M_R(c)\equiv 3\pmod 8$ (éxito por \eqref{eq:norm-units-minus8}).
Si $m\ge 2$, $R^2\equiv 0\pmod 8$, luego $M_R(c)\equiv 7\pmod 8$ (fallo). Esta es precisamente la razón por la cual
una prueba “en una sola frase” aquí es frágil: hay que separar $m=1$ de $m\ge 2$.

\smallskip
\noindent\underline{(2) Capas $t\ge 1$ ( $c$ par ).}
Escribimos $c=2c_1$ y observamos la identidad exacta
\[
M_R(c)=R^2-c^2=4\Big((2^{m-1}r)^2-c_1^2\Big)=4\,M_{R/2}(c_1).
\]
Como $4$ es norma (de hecho potencia de la norma de $\sqrt{-2}$), la pertenencia de $M_R(c)$ al conjunto de normas depende únicamente de
$M_{R/2}(c_1)$.
Además, la masa de la condición $t\ge 1$ es $1/2$, y condicionada a ella, $c_1$ es Haar-uniforme en $\mathbb{Z}_2$.
Por tanto obtenemos la recurrencia exacta
\[
\sigma_{2,-8}^{(m)}=\frac12\cdot \mathbf{1}_{\{\text{éxito en }t=0\}}+\frac12\,\sigma_{2,-8}^{(m-1)}.
\]
Con la evaluación explícita del caso $t=0$ anterior, se verifica que la solución cerrada es
\[
\sigma_{2,-8}^{(m)}=\frac{3}{2^{m+1}},\qquad m\ge 1,
\]
y coincide con \eqref{eq:sigma2-minus8-formula}.
\end{proof}

% ================================================================
\subsection{Expansión exacta vía la zeta de primos y reducción a valores de $\zeta$}
\label{subsec:Aodd-expansion}
% ================================================================

\begin{definition}[Producto impar modelo $A_{\mathrm{odd}}$]
\label{def:Aodd}
Definimos
\[
A_{\mathrm{odd}}=\prod_{p\ge 3}\Big(1-\frac{1}{p(p+1)}\Big).
\]
Este producto aparece como un prototipo de “factor impar” en modelos de primer/segundo orden,
y es un objeto independiente bien definido cuyo análisis espectral es completamente explícito.
\end{definition}

Tomando logaritmos y expandiendo,
\[
\log A_{\mathrm{odd}}
=\sum_{p\ge 3}\log\Big(1-\frac{1}{p(p+1)}\Big)
=-\sum_{n\ge 1}\frac{1}{n}\sum_{p\ge 3}\frac{1}{p^n(p+1)^n}.
\]
Para reescribir $\frac{1}{(p+1)^n}$ usamos la expansión binomial
\[
\frac{1}{(p+1)^n}=p^{-n}\Big(1+\frac{1}{p}\Big)^{-n}
=p^{-n}\sum_{j\ge 0}(-1)^j\binom{n+j-1}{j}\,p^{-j},
\]
válida absolutamente para $p\ge 3$. Sustituyendo y reordenando (justificado por convergencia absoluta),
\begin{equation}\label{eq:logAodd-primezeta}
\log A_{\mathrm{odd}}
=-\sum_{n\ge 1}\frac{1}{n}\sum_{j\ge 0}(-1)^j\binom{n+j-1}{j}
\sum_{p\ge 3}\frac{1}{p^{2n+j}}.
\end{equation}
Introduciendo la zeta de primos $P(s)=\sum_{p}p^{-s}$ obtenemos
\[
\sum_{p\ge 3}\frac{1}{p^{s}}=P(s)-2^{-s}.
\]
Así,
\begin{equation}\label{eq:logAodd-P}
\boxed{
\log A_{\mathrm{odd}}
=-\sum_{n\ge 1}\frac{1}{n}\sum_{j\ge 0}(-1)^j\binom{n+j-1}{j}
\big(P(2n+j)-2^{-(2n+j)}\big).
}
\end{equation}

Finalmente, usamos la identidad clásica (inversión de Möbius) que expresa $P(s)$ en términos de $\zeta$:
\[
P(s)=\sum_{m\ge 1}\frac{\mu(m)}{m}\log\zeta(ms)\qquad(\Re s>1).
\]
Sustituyendo en \eqref{eq:logAodd-P} y exponenciando, se obtiene una factorización puramente en valores de $\zeta$.
El término $-2^{-s}$ aporta exactamente la constante de exclusión del primo $2$, igual a $\log(6/5)$:
\begin{equation}\label{eq:Aodd-zeta-factor}
\boxed{
A_{\mathrm{odd}}=\frac65\prod_{k\ge 2}\zeta(k)^{-b_k},
}
\end{equation}
donde los exponentes $b_k$ vienen dados explícitamente por
\[
b_k=\sum_{\substack{m\ge 1\\ m\mid k}}\frac{\mu(m)}{m}\,a_{k/m},
\qquad
a_\ell:=\sum_{\substack{n\ge 1,\ j\ge 0\\ 2n+j=\ell}}
\frac{(-1)^j}{n}\binom{n+j-1}{j}.
\]
En particular, dado que $\log\zeta(k)\sim 2^{-k}$ y $b_k$ no crece exponencialmente rápido, la serie converge a gran velocidad, lo cual explica la utilidad numérica de \eqref{eq:Aodd-zeta-factor}.

% ================================================================
\subsection{El lugar $2$ en el caso gaussiano: $\Delta=-4$ ($K=\mathbb{Q}(i)$)}
\label{subsec:sigma2-minus4}
% ================================================================

En esta sección cerramos el análisis $2$-ádico del caso gaussiano $K=\mathbb{Q}(i)$, donde
la norma viene dada por
\[
N(a+bi)=a^2+b^2.
\]
A diferencia del caso \emph{inerte}, en $\mathbb{Q}(i)$ no aparece una restricción de \emph{paridad} sobre $v_2(\cdot)$.
Sin embargo, sí aparece una restricción fina sobre la \emph{unidad} impar (clase módulo $4$), que determina la densidad local.

\subsubsection{Caracterización exacta de normas en $\mathbb{Z}_2[i]$}

\begin{lemma}[{Normas integrales en $\mathbb{Z}_2[i]$}]
\label{lem:normZ2i}
Sea $n\in\mathbb{Z}_2$. Entonces $n\in N(\mathbb{Z}_2[i])$ si y solo si $n=0$ o bien
\[
n=2^k u \qquad (k\ge 0,\ u\in\mathbb{Z}_2^\times)
\]
donde la unidad impar satisface
\[
u\equiv 1\pmod 4.
\]
En particular, si $n$ es impar, entonces
\[
n\in N(\mathbb{Z}_2[i])\quad\Longleftrightarrow\quad n\equiv 1\pmod 4.
\]
\end{lemma}

\begin{proof}
\emph{(Necesidad).} Si $n=a^2+b^2$ con $a,b\in\mathbb{Z}_2$, entonces módulo $4$ los cuadrados son $0$ o $1$.
Si $n$ es impar, necesariamente exactamente uno de $a,b$ es impar, luego $n\equiv 1\pmod 4$.
Si $n$ es par, escribimos $n=2^k u$ con $u$ impar. Como $2=N(1+i)$, extraer potencias de $2$ no cambia la pertenencia al conjunto de normas:
$n$ es norma si y solo si $u$ lo es. Por el argumento anterior, si $u$ es norma (impar) entonces $u\equiv 1\pmod 4$.

\smallskip
\emph{(Suficiencia).} Sea $u\in\mathbb{Z}_2^\times$ con $u\equiv 1\pmod 4$. Entonces $u\equiv 1$ o $5\pmod 8$.
En ambos casos hay una representación módulo $8$:
\[
1\equiv 1^2+0^2\pmod 8,\qquad 5\equiv 1^2+2^2\pmod 8.
\]
Estas soluciones se levantan a soluciones en $\mathbb{Z}_2$ por un argumento estándar de lifting, lo que da $u=a^2+b^2$ en $\mathbb{Z}_2$.
Finalmente, como $2=N(1+i)$, toda potencia $2^k u$ también es norma.
\end{proof}

\subsubsection{Densidad local para la familia $F_R(d)=(R-d)(R+d)$}

Recordemos $M_R(c)=R^2-c^2=(R-c)(R+c)$, y definimos la densidad local
\[
\sigma_{2,-4}^{(m)}:=\mathbb{P}_{c\in\mathbb{Z}_2}\Big(M_R(c)\in N(\mathbb{Z}_2[i])\cup\{0\}\Big),
\qquad m=v_2(R).
\]

\begin{theorem}[Densidad $2$-ádica cerrada para $\Delta=-4$]
\label{thm:sigma2-minus4}
Sea $R\in\mathbb{Z}$ y $m=v_2(R)$. Entonces
\begin{equation}\label{eq:sigma2-minus4-formula}
\boxed{
\sigma_{2,-4}^{(m)}=\frac{3}{2^{m+2}}.
}
\end{equation}
En particular, en el régimen estándar de $R$ impar ($m=0$),
\[
\boxed{\sigma_{2,-4}^{(0)}=\frac34.}
\]
\end{theorem}

\begin{proof}
Escribimos $R=2^m r$ con $r$ impar.

\smallskip
\noindent\textbf{Paso 1: si $v_2(c)=t<m$, entonces $M_R(c)$ no es norma.}
Escribimos $c=2^t u$ con $u$ impar y $t<m$. Entonces
\[
M_R(c)=R^2-c^2
=2^{2t}\Big(2^{2(m-t)}r^2-u^2\Big).
\]
Como $m-t\ge 1$, se tiene $2^{2(m-t)}r^2\equiv 0\pmod 4$, mientras que $u^2\equiv 1\pmod 4$,
luego
\[
2^{2(m-t)}r^2-u^2\equiv -1\equiv 3\pmod 4.
\]
Por el Lema~\ref{lem:normZ2i}, una unidad impar $\equiv 3\pmod 4$ no es norma en $\mathbb{Z}_2[i]$.
Como $2^{2t}$ sí es norma, concluimos que $M_R(c)$ no puede ser norma.
Por tanto, el evento ``$M_R(c)$ es norma'' implica necesariamente $v_2(c)\ge m$.

\smallskip
\noindent\textbf{Paso 2: reducción a $m=0$.}
Condicionado a $v_2(c)\ge m$, escribimos $c=2^m x$ con $x\in\mathbb{Z}_2$ Haar-uniforme.
Entonces
\[
M_R(c)=2^{2m}(r^2-x^2).
\]
Como $2^{2m}=N((1+i)^{2m})$ es norma, por el Lema~\ref{lem:normZ2i} se tiene que
$M_R(c)$ es norma si y solo si $r^2-x^2$ es norma.
Por tanto
\[
\sigma_{2,-4}^{(m)}
=\mathbb{P}(v_2(c)\ge m)\cdot \mathbb{P}_{x\in\mathbb{Z}_2}(r^2-x^2\in N(\mathbb{Z}_2[i])\cup\{0\}).
\]
Además, $\mathbb{P}(v_2(c)\ge m)=2^{-m}$.

\smallskip
\noindent\textbf{Paso 3: cálculo base ($m=0$), probabilidad $3/4$.}
Sea $r$ impar fijo y $x\in\mathbb{Z}_2$ Haar-uniforme.

\smallskip
\noindent\underline{(i) $x$ par.} Entonces $x^2\equiv 0\pmod 4$ y $r^2\equiv 1\pmod 4$,
luego $r^2-x^2\equiv 1\pmod 4$ y por el Lema~\ref{lem:normZ2i} es norma. Contribución: $\mathbb{P}(x\ \text{par})=1/2$.

\smallskip
\noindent\underline{(ii) $x$ impar.} Entonces $r\pm x$ son pares y
\[
r^2-x^2=(r-x)(r+x)=4ab,\qquad a=\frac{r-x}{2},\ \ b=\frac{r+x}{2}.
\]
Como $a+b=r$ es impar, exactamente uno de $a,b$ es par y el otro impar.
Sea $E$ el factor par y $O$ el factor impar, de modo que $ab=EO$ y $O$ es unidad impar.
La condición de ser norma depende únicamente de la unidad impar asociada a $EO$ tras extraer su potencia de $2$.
Condicionado a $x$ impar, la variable $E$ es Haar-uniforme en $2\mathbb{Z}_2$.
La aplicación $x \mapsto \text{unidad}(r-x)$ es una biyección que preserva la medida de Haar localmente (salvo factores de escala que se cancelan al normalizar).
Por tanto $E/2$ es Haar-uniforme en $\mathbb{Z}_2$ y su parte impar (unidad módulo $4$) es equiprobable en $\{1,3\}$.
Multiplicar por la unidad $O$ permuta $\{1,3\}$, así que la probabilidad de que la unidad resultante sea $\equiv 1\pmod 4$
es exactamente $1/2$. Contribución: $\mathbb{P}(x\ \text{impar})\cdot \frac12=\frac12\cdot\frac12=\frac14$.

\smallskip
Sumando (i)+(ii), obtenemos
\[
\mathbb{P}_{x\in\mathbb{Z}_2}(r^2-x^2\ \text{norma})=\frac12+\frac14=\frac34.
\]

\smallskip
\noindent\textbf{Paso 4: conclusión.}
\[
\sigma_{2,-4}^{(m)}=2^{-m}\cdot\frac34=\frac{3}{2^{m+2}}.
\]
\end{proof}

\begin{remark}[Lectura conceptual]
En el caso inerte ($\Delta\equiv 5\pmod 8$) la obstrucción es una \emph{paridad} de $v_2$, y la densidad mejora al aumentar $m=v_2(R)$.
En el caso gaussiano ($\Delta=-4$), la obstrucción es una \emph{clase de unidad} ($\equiv 1\pmod 4$), y el evento obliga a que $d$
comparta al menos $m$ potencias de $2$ con $R$, lo cual introduce el factor geométrico $2^{-m}$.
\end{remark}

% ================================================================
\subsection{El lugar $2$ en el caso eisensteiniano: $\Delta=-3$ ($K=\mathbb{Q}(\sqrt{-3})$)}
\label{subsec:sigma2-minus3}
% ================================================================

En este apartado analizamos la contribución local en $p=2$ para el campo
$K=\mathbb{Q}(\sqrt{-3})$, cuyo anillo de enteros es $\mathcal{O}_K=\mathbb{Z}[\omega]$,
$\omega=\frac{-1+\sqrt{-3}}{2}$, con norma $N(a+b\omega)=a^2-ab+b^2$.
En particular, el discriminante fundamental es $\Delta=-3\equiv 5\pmod 8$, lo cual implica que
el primo $2$ es \emph{inerte} y la extensión local $K_2/\mathbb{Q}_2$ es cuadrática \emph{no ramificada}.

\subsubsection{Grupo de normas en una extensión cuadrática no ramificada}

\begin{lemma}[Normas en extensiones cuadráticas no ramificadas]
\label{lem:norm-unramified-2}
Sea $L/\mathbb{Q}_2$ una extensión no ramificada de grado $2$.
Entonces
\[
N(L^\times)=\{x\in\mathbb{Q}_2^\times:\ v_2(x)\ \text{es par}\}.
\]
En particular, toda unidad $u\in\mathbb{Z}_2^\times$ es norma, y la única obstrucción es la paridad de $v_2(x)$.
\end{lemma}

\begin{proof}
Sea $v_2$ la valuación en $\mathbb{Q}_2$ y $v_L$ la de $L$.
Para $y\in L^\times$ se cumple $v_2(N(y))=[L:\mathbb{Q}_2]\,v_L(y)=2v_L(y)$, luego $v_2(N(y))$ es par.
Recíprocamente, si $x=2^{2k}u$ con $u\in\mathbb{Z}_2^\times$, basta ver que $u$ es norma:
el morfismo norma sobre residuos $\mathbb{F}_{4}^\times\to \mathbb{F}_2^\times$ es sobreyectivo, luego las unidades son normas
y por lifting se obtiene $u=N(y)$ con $y\in\mathcal{O}_L^\times$.
Finalmente, $2^{2k}=N(2^k)$, así que $x$ es norma.
\end{proof}

Definimos, como antes,
\[
\sigma_{2,-3}^{(m)}:=\mathbb{P}_{c\in\mathbb{Z}_2}\big(M_R(c)=(R-c)(R+c)\in N(K_2^\times)\cup\{0\}\big),
\qquad m=v_2(R).
\]

\begin{theorem}[Densidad local $2$-ádica en el caso inerte: $\Delta=-3$]
\label{thm:sigma2-minus3}
Sea $R\in\mathbb{Z}$, $m=v_2(R)$, y $d$ Haar-uniforme en $\mathbb{Z}_2$.
Entonces
\begin{equation}\label{eq:sigma2-minus3-formula}
\boxed{
\sigma_{2,-3}^{(m)}=1-\frac{1}{3\cdot 2^{m}}.
}
\end{equation}
En particular, para $R$ impar ($m=0$),
\[
\boxed{\sigma_{2,-3}^{(0)}=\frac{2}{3}.}
\]
\end{theorem}

\begin{proof}
Es exactamente el Teorema~\ref{thm:sigma2-inerte}, ya que en $\Delta=-3$ el lugar $2$ es no ramificado y la condición local es
paridad de $v_2$.
\end{proof}

% ================================================================
\subsection{El lugar $3$ en el caso Eisenstein: $\Delta=-3$ (primo ramificado)}
\label{subsec:sigma3-minus3}
% ================================================================

En el campo $K=\mathbb{Q}(\sqrt{-3})$ el primo $3$ es ramificado (pues $3\mid\Delta$).
La completación local $K_3=\mathbb{Q}_3(\sqrt{-3})$ es una extensión cuadrática \emph{totalmente ramificada} de $\mathbb{Q}_3$.
En este caso, la obstrucción para ser norma no es una paridad de valuación, sino una condición sobre la \emph{clase de la unidad} módulo $3$.

\subsubsection{Caracterización del grupo de normas en $K_3/\mathbb{Q}_3$}

\begin{lemma}[Normas en una extensión cuadrática totalmente ramificada, $p$ impar]
\label{lem:norm-ramified-odd}
Sea $p$ impar y $L/\mathbb{Q}_p$ una extensión cuadrática totalmente ramificada.
Entonces:
\begin{enumerate}[leftmargin=2em]
\item La norma es sobreyectiva sobre la valuación: para todo $k\in\mathbb{Z}$ existe $y\in L^\times$ tal que
$v_p(N(y))=k$.
\item Para unidades, la imagen cumple
\[
N(\mathcal{O}_L^\times)\equiv (\mathbb{F}_p^\times)^2 \pmod p.
\]
En particular, para $p=3$ se tiene $(\mathbb{F}_3^\times)^2=\{1\}$ y por tanto:
\begin{equation}\label{eq:norm-units-mod3}
u\in N(L^\times)\cap \mathbb{Z}_3^\times
\quad\Longleftrightarrow\quad
u\equiv 1\pmod 3.
\end{equation}
\end{enumerate}
\end{lemma}

\begin{proof}
Como $L/\mathbb{Q}_p$ es totalmente ramificada de grado $2$, el grado residual es $f=1$.
Para un uniformizante $\pi_L$ se sabe que $v_p(N(\pi_L))=f=1$, lo que implica que las valuaciones de normas recorren todo $\mathbb{Z}$.
Para unidades $u\in\mathcal{O}_L^\times$, la conjugación es trivial en el cuerpo residual (pues $f=1$),
luego $N(u)=u\cdot \bar u\equiv u^2\pmod p$, de donde la imagen módulo $p$ son exactamente los cuadrados.
Para $p=3$ esto fuerza \eqref{eq:norm-units-mod3}.
\end{proof}

Aplicando este lema a $L=K_3=\mathbb{Q}_3(\sqrt{-3})$ obtenemos que,
para $x\in\mathbb{Q}_3^\times$ escrito como $x=3^k u$ con $u\in\mathbb{Z}_3^\times$,
\begin{equation}\label{eq:norm-criterion-3}
x\in N(K_3^\times)
\quad\Longleftrightarrow\quad
u\equiv 1\pmod 3.
\end{equation}

\subsubsection{Densidad local para la familia $F_R(d)=R^2-d^2$}

Definimos
\[
\sigma_{3,-3}^{(m)}:=\mathbb{P}_{d\in\mathbb{Z}_3}\Big(F_R(d)=R^2-d^2\in N(K_3^\times)\cup\{0\}\Big),
\qquad m=v_3(R).
\]

\begin{theorem}[Densidad local $3$-ádica para $\Delta=-3$]
\label{thm:sigma3-minus3}
Sea $R\in\mathbb{Z}$, $m=v_3(R)$ y $d$ Haar-uniforme en $\mathbb{Z}_3$.
Entonces:
\begin{equation}\label{eq:sigma3-minus3-formula}
\boxed{
\sigma_{3,-3}^{(m)}=\frac{2}{3^{m+1}}.
}
\end{equation}
En particular, para el régimen base $3\nmid R$ ($m=0$),
\[
\boxed{\sigma_{3,-3}^{(0)}=\frac{2}{3}.}
\]
\end{theorem}

\begin{proof}
Escribimos $R=3^m r$ con $r\in\mathbb{Z}_3^\times$.
Dividimos según $t=v_3(d)$.

\smallskip
\noindent\textbf{(1) Si $t<m$, hay fallo seguro.}
Escribimos $d=3^t u$ con $u\in\mathbb{Z}_3^\times$ y $t<m$.
Entonces
\[
F_R(d)=R^2-d^2=3^{2t}\big(3^{2(m-t)}r^2-u^2\big).
\]
Como $m-t\ge 1$, el término $3^{2(m-t)}r^2\equiv 0\pmod 3$, mientras que $u^2\equiv 1\pmod 3$,
luego la unidad residual de $F_R(d)$ es
\[
3^{2(m-t)}r^2-u^2 \equiv -1 \equiv 2 \pmod 3,
\]
que no satisface \eqref{eq:norm-criterion-3}. Por tanto no es norma.

\smallskip
\noindent\textbf{(2) Si $t>m$, hay éxito seguro.}
Si $t>m$, entonces $d$ es múltiplo de $3^{m+1}$ y
\[
F_R(d)=R^2-d^2 \equiv R^2 \equiv 3^{2m} r^2 \pmod{3^{2m+1}}.
\]
En particular $v_3(F_R(d))=2m$ y la unidad residual es $r^2\equiv 1\pmod 3$, que sí es norma.

\smallskip
\noindent\textbf{(3) Capa crítica $t=m$.}
Escribimos $d=3^m u$ con $u\in\mathbb{Z}_3^\times$. Entonces
\[
F_R(d)=3^{2m}(r^2-u^2).
\]
Como $r,u$ son unidades, $r^2\equiv u^2\equiv 1\pmod 3$, luego $r^2-u^2\equiv 0\pmod 3$ y
$v_3(r^2-u^2)\ge 1$. El criterio de norma \eqref{eq:norm-criterion-3} depende de la unidad residual tras extraer la potencia exacta de $3$.
Para computar la probabilidad, basta trabajar módulo $9$:
los cuadrados de las $6$ clases unidades módulo $9$ toman exactamente los valores $\{1,4,7\}$, cada uno con probabilidad $1/3$.
Fijado $r^2\ (\mathrm{mod}\ 9)$, para $u$ unidad al azar:
\begin{itemize}[leftmargin=2em]
\item con probabilidad $1/3$ se tiene $u^2\equiv r^2\ (\mathrm{mod}\ 9)$, y entonces $r^2-u^2\equiv 0\ (\mathrm{mod}\ 9)$, es decir $v_3(r^2-u^2)\ge 2$;
\item con probabilidad $1/3$, $r^2-u^2\equiv 3\ (\mathrm{mod}\ 9)$, y entonces $(r^2-u^2)/3\equiv 1\ (\mathrm{mod}\ 3)$ (éxito);
\item con probabilidad $1/3$, $r^2-u^2\equiv 6\ (\mathrm{mod}\ 9)$, y entonces $(r^2-u^2)/3\equiv 2\ (\mathrm{mod}\ 3)$ (fallo).
\end{itemize}
En la rama $v_3(r^2-u^2)\ge 2$, al refinar a niveles superiores la unidad residual tras extraer la valuación exacta
es equidistribuida entre $\{1,2\}$, de modo que contribuye con probabilidad $1/2$ al éxito.
Así, condicionado a $t=m$, la probabilidad total de éxito es
\[
\frac{1}{3}\cdot 1 \;+\;\frac{1}{3}\cdot 0 \;+\;\frac{1}{3}\cdot \frac{1}{2} \;=\;\frac{1}{2}.
\]
Combinando con la probabilidad de acceso a la capa crítica:
\[
\Prob(\text{éxito}\mid v_3(d)\ge m)=\frac23\cdot\frac12+\frac13\cdot 1=\frac23.
\]

\smallskip
Finalmente, por (1) y (2) vemos que solo contribuyen los $d$ con $v_3(d)\ge m$,
y esa condición tiene medida $3^{-m}$. En ese subconjunto, el cálculo base produce éxito con probabilidad $2/3$:
\[
\sigma_{3,-3}^{(m)}=3^{-m}\cdot \frac{2}{3}=\frac{2}{3^{m+1}}.
\]
\end{proof}

% ================================================================
\section{Diagnóstico de residuales del ajuste de cola}
\label{app:residuals}
% ================================================================

Las siguientes figuras demuestran que la estimación de $\Csinf(\Delta)$ no es un artefacto de regresión.
Los residuales del ajuste de cola muestran comportamiento estacionario sin tendencias sistemáticas.

\begin{figure}[htbp]
\centering
\begin{subfigure}[b]{0.32\linewidth}
\incfig[width=\linewidth]{../../Figuras/Articulo/D-4_tail_residual.pdf}
\caption{$\Delta=-4$}
\end{subfigure}
\hfill
\begin{subfigure}[b]{0.32\linewidth}
\incfig[width=\linewidth]{../../Figuras/Articulo/D-7_tail_residual.pdf}
\caption{$\Delta=-7$}
\end{subfigure}
\hfill
\begin{subfigure}[b]{0.32\linewidth}
\incfig[width=\linewidth]{../../Figuras/Articulo/D-43_tail_residual.pdf}
\caption{$\Delta=-43$}
\end{subfigure}
\caption{Residuales del ajuste de cola para discriminantes representativos. La ausencia de tendencia confirma la validez del modelo $C_{\sigma,\Delta}(R)\approx C_\infty+a/\log R$.}
\label{fig:tail-residuals}
\end{figure}

% ================================================================
\section{Figuras adicionales: normalización y colapso}
\label{app:additional-figs}
% ================================================================

\subsection{Normalización raw vs normalizado adicional}
Las siguientes figuras muestran el efecto de normalización para discriminantes adicionales, complementando la \Cref{fig:D43-raw-vs-norm}.

\begin{figure}[htbp]
\centering
\begin{subfigure}[b]{0.48\linewidth}
\incfig[width=\linewidth]{../../Figuras/Articulo/D-4_raw_vs_norm.pdf}
\caption{$\Delta=-4$ (gaussiano)}
\end{subfigure}
\hfill
\begin{subfigure}[b]{0.48\linewidth}
\incfig[width=\linewidth]{../../Figuras/Articulo/D-7_raw_vs_norm.pdf}
\caption{$\Delta=-7$ (outlier)}
\end{subfigure}
\caption{Efecto de normalización por $H_\Delta(R)$ para discriminantes adicionales. Arriba: observable crudo oscilando. Abajo: observable estabilizado.}
\label{fig:raw-vs-norm-additional}
\end{figure}

\subsection{Colapso lineal adicional}
Las siguientes figuras muestran el colapso $C_{\rm raw}$ vs $H_\Delta$ para discriminantes adicionales, complementando la \Cref{fig:D43-collapse}.

\begin{figure}[htbp]
\centering
\begin{subfigure}[b]{0.48\linewidth}
\incfig[width=\linewidth]{../../Figuras/Articulo/D-4_Craw_vs_H.pdf}
\caption{$\Delta=-4$ (gaussiano)}
\end{subfigure}
\hfill
\begin{subfigure}[b]{0.48\linewidth}
\incfig[width=\linewidth]{../../Figuras/Articulo/D-7_Craw_vs_H.pdf}
\caption{$\Delta=-7$ (outlier)}
\end{subfigure}
\caption{Colapso lineal $C_{\rm raw}(R)$ vs $H_\Delta(R)$ para discriminantes adicionales. El alineamiento confirma la estructura inerte en todos los casos.}
\label{fig:collapse-additional}
\end{figure}

% ================================================================
% ================================================================
\section{Conclusiones y extensiones}
\label{sec:conclusiones}
% ================================================================

El presente trabajo establece de manera incondicional la estructura local fina del conteo de normas,
identifica el mecanismo de ``anclaje inerte'' y proporciona fórmulas cerradas para las densidades $p$-ádicas
(Teoremas~\ref{thm:local-odd} y \ref{thm:sigma2-inerte}).
La constante global queda completamente determinada por la interacción entre el residuo de la zeta de Dedekind
y la serie singular local:
\[
\Csinf(\Delta)=2e^{-\euler}L(1,\chi_\Delta)\,\kappa_2(\Delta)\prod_{p\ge 3}\kappa_p(\Delta).
\]

La elevación de la Conjetura~\ref{conj:global} a teorema incondicional depende exclusivamente de probar la hipótesis
de distribución $\mathcal D_\theta$ para la familia cuadrática $R^2-c^2$ en progresiones aritméticas.
Ese punto queda aislado en el Teorema~\ref{thm:global-conditional}: una vez disponible dicho input analítico,
la asintótica global se activa inmediatamente con la constante canónica ya fijada por la teoría local.

Aunque el análisis se centra en $R^2-c^2$, el principio de capa crítica sugiere una extensión natural a familias
descomponibles $Q(x,y)=L_1(x,y)L_2(x,y)$ donde una relación aditiva rígida controle la paridad de valuaciones
en primos inertes.

Como agenda inmediata, el marco desarrollado aquí sugiere cuatro direcciones de trabajo concretas:
\begin{enumerate}
\item Extender el anclaje inerte a familias cuadráticas más generales $f(c)$, identificando el rol del discriminante local en la capa crítica.
\item Obtener una versión incondicional global del Teorema~\ref{thm:global-conditional} bajo hipótesis analíticas estándar (por ejemplo, GRH o un nivel de distribución óptimo).
\item Tratar sistemáticamente el régimen $h(\Delta)>1$, incorporando clases de ideales y su contribución a la constante singular.
\item Explorar análogos en órdenes no maximales y contextos aritmético-geométricos afines donde aparezcan normas con estructura local anclada.
\end{enumerate}

% ================================================================
\section{Bibliografía}
\begin{thebibliography}{9}

\bibitem{Apostol}
T.~M. Apostol.
\newblock \emph{Introduction to Analytic Number Theory}.
\newblock Springer, 1976.

\bibitem{IwaniecKowalski}
H.~Iwaniec and E.~Kowalski.
\newblock \emph{Analytic Number Theory}.
\newblock AMS Colloquium Publications, 2004.

\bibitem{HalberstamRichert}
H.~Halberstam and H.-E.~Richert.
\newblock \emph{Sieve Methods}.
\newblock Academic Press, 1974.

\bibitem{FriedlanderIwaniec}
J.~Friedlander and H.~Iwaniec.
\newblock \emph{Opera de Cribro}.
\newblock AMS Colloquium Publications, 2010.

\bibitem{Henriot2012}
K.~Henriot.
\newblock \emph{Nair--Tenenbaum bounds uniform with respect to the discriminant}.
\newblock Math. Proc. Cambridge Philos. Soc., 152(3):405--424, 2012.

\bibitem{Neukirch}
J.~Neukirch.
\newblock \emph{Algebraic Number Theory}.
\newblock Springer, 1999.

\end{thebibliography}

\end{document}
